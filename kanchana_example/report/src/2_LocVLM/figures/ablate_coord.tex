\begin{table}[t]
\centering
\small
\def\arraystretch{1.1}  % height
\setlength\tabcolsep{0.9em}  % width
\scalebox{0.97}{
\begin{tabular}{l|c|c|c}
\toprule
CR   & GQA (Acc)  & RD (METEOR)  & A-QA (Acc)\\ \midrule
NFP  & 46.1 & 19.6 & 37.1 \\ 
\rowcolor{Gray}
IVB  & 47.3 & 20.7 & 37.4 \\ 
DIGA & 47.0 & 20.8 & 37.3 \\ \bottomrule
\end{tabular}
}
\caption[Ablation on Coordinate Representation]{Ablation on Coordinate Representation (CR) methods: we compare each of the three proposed CR variants, namely normalized floating point values (NFP), integer valued binning (IVB), and deviation from image-grid based anchors (DIGA).}
\label{locvlm_tbl:ablate_coord}
\end{table}