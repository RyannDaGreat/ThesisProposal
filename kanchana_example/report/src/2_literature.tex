\chapter{Literature Review}
\label{chapter:literature}

Recent advances in video understanding have evolved through several interconnected directions. Early video self-supervised learning (SSL) methods relied on visual-only pretext tasks such as frame prediction, temporal shuffling, and future generation, enabling representation learning without labels. These were followed by contrastive methods and masked autoencoders, which improved performance by leveraging better view sampling and reconstruction objectives. A parallel thread explored adapting image-based CLIP models to the video domain, either by extending frame-wise embeddings or fine-tuning with video data. While effective in supervised scenarios, many of these adaptations showed limited gains in zero-shot settings, motivating the need for more generalizable, language-aligned representations. In tandem, image-text foundation models like CLIP and ALIGN drove a broader shift toward multimodal learning, enabling zero-shot classification by aligning visual inputs with natural language supervision. Some approaches extend this idea to videos by embedding temporal information or using action descriptions as proxies for labels.

Building on this progression, the rise of large language models (LLMs) has catalyzed a new wave of vision-language research focused on more flexible, instruction-following systems. Generative visual-language models (V-LLMs) such as LLaVA and Video-ChatGPT combine CLIP-like visual encoders with LLMs to support open-ended tasks like captioning, reasoning, and question answering. However, these models often exhibit poor spatial grounding, leading to growing interest in enhancing spatial reasoning. Recent methods address this through textual coordinate prompts, region-level tokenization, or spatially-aware instruction tuning. In the video domain, long-form question answering has emerged as a benchmark for temporal and causal understanding, with approaches incorporating object-centric representations, efficient video encoding, and LLM-based reasoning. Meanwhile, diffusion models have gained traction for generating pixel-level motion—such as optical flow or trajectories—further bridging perception and action. These trends converge in robotics, where vision-language models are increasingly used for language-conditioned control, leveraging general-purpose pretrained representations with minimal task-specific data.

Motivated by these converging trends, our work investigates unifying video representation learning with natural language based systems. By leveraging the compositional and interpretable nature of language, we align video learning with human like abstraction and reasoning. We further incorporate spatial awareness by focusing on object-centric cues such as locations and motion to enhance the grounding and interpretability of video-language models. Finally, we explore how motion itself can be modeled as a structured, language-conditioned representation under self-supervised settings, enabling more generalizable and semantically aligned video understanding as well as real-world interaction.

\section{Language Based Video Self-Supervised Learning}
\label{lss:related}

\bhdr{Self-Supervised Learning in Videos} was initially dominated by pretext tasks specific to the video domain \cite{mathieu2015deep, PatrauceanHC16, walker2016uncertain, pmlr-v37-srivastava15, Vondrick16a, vondrick2018tracking, Agrawal_2015_ICCV, Goroshin_2015_ICCV, DBLP:journals/corr/IsolaZKA15, Misra-2016-5596, 7410677, piergiovanni2020evolving}. Recently a shift to contrastive losses led to \cite{Feichtenhofer_large, han2019video, han2020self, qian2020spatiotemporal, hjelm2020representation, recasens2021broaden} with some variants focused on video specific view generation \cite{Huang_2021_ICCV, chen2021rspnet, Behrmann2021LongSV, Dave2021TCLRTC, Ran2021SVT}. An alternate direction has been masked auto-encoders \cite{Tong2022VidMAE}.
To the best of our knowledge, existing video self-supervised learning (SSL) approaches operate purely within the visual domain. By video SSL, we refer to methods that utilize only videos with no paired captions (or labels) for each video during training.
In contrast, our proposed LSS learns purely from videos in a self-supervised manner, integrating pre-trained language-image models to learn language aligned representations. 

\vspace{0.5em}


\bhdr{Zero-shot Action Recognition} began with manual attribute and feature selection \cite{liu2011recognizing,zellers2017zero, jain2015objects2action,gao2019know,gao2020learning} with later works utilizing action word embeddings \cite{brattoli2020rethinking,xu2017transductive}. The idea of connecting action categories with elaborate descriptions of those actions, within language embedding spaces \cite{chen2021erzsar,zhu2018ur} has been a next step and is closely related to our work. This idea is also explored in image domain to boost zero-shot performance \cite{menon2022visual}. While our work is inspired by such approaches, in contrast, we use relations between such actions and descriptions as self-supervised signals for learning. 
Recent image CLIP models \cite{radford2021clip,jia2021align} are another line of works achieving strong performance on some video classification tasks, with only single frame processing. Multiple approaches build on image CLIP \cite{radford2021clip} to learn video level representations \cite{wang2021actionclip, luo2022clip4clip, bain2022cliphitchhiker, lin2022evl, ma2022xclip, Kahatapitiya2023VicTRVT} under fully-supervised settings. While achieving strong performance on the training datasets, their zero-shot improvements over CLIP are minimal or even subpar (see \cref{lss_tbl:zeroshot}). Therein, LSS focuses on zero-shot performance under self-supervised settings while retaining (and improving) the generality of the representation space.

\vspace{0.5em}

\bhdr{Self-training} methods leverage pseudo-labels on unlabeled data \cite{mean_teacher,fixmatch,remixmatch} for supervised-fashion training. Recently they have been combined with CLIP models for zero-shot operation \cite{Li2022MaskedUS, Kahana2022ImprovingZM}. While inspired by such self-training approaches, our proposed LSS differs in its continuous feature space self-distillation, language-based relations enforcing, video domain operation, and cross-dataset transfer for zero-shot operation. 

\vspace{0.5em}

\bhdr{Adapting image-CLIP models to video} under fully-supervised settings has gathered much interest \cite{xue2022clipvip, yan2022videococa, qian2022multimodal, ju2022prompting, rasheed2022fine, cheng2022vindlu}. Expanding backbones for temporal modeling, multi-modal fusion, secondary training objectives, partial parameter updates, and scaling-up data are key ideas explored \cite{Kahatapitiya2023VicTRVT,cheng2022vindlu}. In contrast, LSS is a first to operate under self-supervised settings using no video annotations. 

\vspace{0.5em}

\bhdr{Contemporary work} in \cite{Lin2023MAtchEA} adapts image CLIP features to video tasks label free similar to our work. ViFi-CLIP \cite{Rasheed2022FinetunedCM} introduces zero-shot action recognition benchmarks and similarly adapts CLIP to videos retaining generality. Using LLMs for action recognition is also explored in \cite{Hanu2023LanguageAT}. 

\section{Spatial Reasoning in Visual-LLMs}
\label{locvlm:related}

\noindent
\textbf{Localization in Contrastive Vision Language Models}:
Foundation vision language models (VLMs) such as CLIP \cite{radford2021learning} resulted in extensive exploration into language-tied localization in images both under dense (pixel / bounding-box) supervision \cite{gu2021vild, kamath2021mdetr, Li2022AdaptingCF, Li2021GroundedLP, Zeng2021MultiGrainedVL,Dou2022CoarsetoFineVP,ghiasi2022open, li2022language, Zhang2022GLIPv2UL,Ding2021DecouplingZS} and weak supervision \cite{xu2022groupvit, Xu2023LearningOS, Zhou2021ExtractFD, yao2022filip,cui2022democratizing, Zhang2023AssociatingSG,Luo2022SegCLIPPA,Ranasinghe2022PerceptualGI,Mukhoti2022OpenVS}. Recovering explicit localization information within model representations has enabled more robust operation for certain tasks \cite{Ranasinghe2022PerceptualGI}. While our work differs from this contrastive setting given our use of LLM based generative predictions, we similarly explore how explicit location information within the language modality can improve V-LLMs.  

\vspace{0.5em}

\noindent
\textbf{Visual Large Language Models (V-LLMs)}:
The advent of powerful large language models (LLMs) such as GPT-3 \cite{brown2020language}, Chat-GPT \cite{gpt4}, and PaLM \cite{chowdhery2022palm}, as well as their open-source counterparts BLOOM~\citep{scao2022bloom}, Vicuna \cite{vicuna2023}, and LLaMA~\citep{touvron2023llama,touvron2023llama2}, has resulted in direct use of these LLMs for computer vision tasks \cite{Gupta2022VisualPC, Suris2023ViperGPTVI}. Alternate lines of work explore how LLMs can be connected to existing visual foundation models \cite{alayrac2022flamingo,anas_awadalla_2023_7733589,Ranasinghe2023LanguagebasedAC,li2023blip,liu2023visual,Maaz2023VideoChatGPT}, in particular to CLIP visual backbones \cite{radford2021learning}. While earlier models explored large-scale (millions to billions of samples) image-text training \cite{alayrac2022flamingo,anas_awadalla_2023_7733589}, later models \cite{li2023blip,liu2023visual,Maaz2023VideoChatGPT} scale down on data dependency. LLaVA \cite{liu2023visual} in particular scales down on pre-training data to under 1 million image-text pairs, and use instruction fine-tuning \cite{Wei2021FinetunedLM} to enable human-style conversation with visual awareness. This is extended to video domain in \cite{Maaz2023VideoChatGPT,ranasinghe2024understanding}. A shortcoming of these models is their lack of spatial awareness or location understanding in image space \cite{chen2023shikra,Gokhale2022BenchmarkingSR,Cho2022DALLEVALPT}.
% - in our work we attempt to address this limitation. 
Spatial reasoning limitations in generative VLMs are studied in \cite{Gokhale2022BenchmarkingSR,Cho2022DALLEVALPT}. Similar failures in captioning (and VQA) models are explored in \cite{Kamath2023WhatsW}. A solution in \cite{Hsu2023WhatsLC} proposes code-generation based reasoning. Our work tackles these same limitations but follows an alternate direction of spatial-aware instruction fine-tuning.
Another line of recent works \cite{zhang2023gpt4roi,zhao2023bubogpt,zang2023contextual,peng2023kosmos,You2023FerretRA,wang2023visionllm,lai2023lisa} tackle this by introducing architectural modifications to explicitly extract region level features that are injected to the LLM as special tokens. While introducing extra tokens and layers, this also separates the localization task from language. 
In contrast, we use a generic architectures with purely textual location information (i.e. image space coordinates as text).  
% We postulate that generic architectures with purely language driven location information (i.e. reasoning in image coordinate space) would inject LLMs with stronger spatial awareness more similar to human visual understanding and reasoning. 
Concurrent work in \cite{chen2023shikra} explores this same idea, but we differ in 3 ways with, a) focus on optimal coordinate representation forms, b) data-efficient pseudo-labelling strategies, and c) video domain operation (see also \cref{tbl:related}). 

\begin{table}[t]
\centering
\small
\def\arraystretch{1.0}  % height
\setlength\tabcolsep{0.5em}  % width
\scalebox{0.95}{
\begin{tabular}{l|c|c|c|c}
\toprule
Method          & Kosmos \cite{peng2023kosmos} & Ferret \cite{You2023FerretRA} & Shikra \cite{chen2023shikra} & Proposed \\ \midrule
Unified Arch.   & \xmark & \xmark & \cmark & \cmark  \\ 
Purely Textual  & \xmark & \xmark & \cmark & \cmark  \\ 
Pseudo Data     & \xmark & \xmark & \xmark & \cmark  \\ 
Video Domain    & \xmark & \xmark & \xmark & \cmark  \\ \bottomrule
\end{tabular}
}
\vspace{-0.5em}
\caption{Related Work Comparison: 
% We highlight key distinctions of our proposed method against three most similar related works. 
A unified architecture, purely textual inputs, pseudo data for scalable learning, and video domain operation distinguishes our proposed work from these prior methods.}
\label{tbl:related}
\end{table}

\vspace{0.5em}
\noindent
\textbf{Location Representations}:
Selecting regions within an image has a rich history in computer vision \cite{malik2001visual,uijlings2013selective} with greater focus on location outputs since the popularity of object detection \cite{girshick2015fastrcnn, redmon2016yolo, tian2019fcos, carion2020detr, chen2022diffusiondet, chen2021pix2seq, wang2023visionllm}. Early anchor-based methods regress locations from anchor centers \cite{girshick2015fastrcnn, redmon2016yolo}, followed by direct location regression from object-level features \cite{tian2019fcos, carion2020detr}. Recent works explore generative location predictions with diffusion processes \cite{chen2022diffusiondet} and sequence-generation \cite{chen2021pix2seq, wang2023visionllm}. Ours resembles the latter given our use of an LLM, next token prediction objective, and sequential generation of textual location representations. However, \cite{chen2021pix2seq, wang2023visionllm} utilize 1000 specialized location tokens (introduced to the LLM vocabulary) corresponding to 1000 bins uniformly spread across image space. While we explore similar binning strategies, in contrast we introduce no additional tokens, focus on purely textual representation of locations, and explore multiple textual location representation forms.  



\begin{table}[t]
\centering
\small
\def\arraystretch{1.0}  % height
\setlength\tabcolsep{0.5em}  % width
\scalebox{0.95}{
\begin{tabular}{l|c|c|c|c}
\toprule
Method          & Kosmos \cite{peng2023kosmos} & Ferret \cite{You2023FerretRA} & Shikra \cite{chen2023shikra} & Proposed \\ \midrule
Unified Arch.   & \xmark & \xmark & \cmark & \cmark  \\ 
Purely Textual  & \xmark & \xmark & \cmark & \cmark  \\ 
Pseudo Data     & \xmark & \xmark & \xmark & \cmark  \\ 
Video Domain    & \xmark & \xmark & \xmark & \cmark  \\ \bottomrule
\end{tabular}
}
\vspace{-0.5em}
\caption{Related Work Comparison: 
% We highlight key distinctions of our proposed method against three most similar related works. 
A unified architecture, purely textual inputs, pseudo data for scalable learning, and video domain operation distinguishes our proposed work from these prior methods.}
\label{tbl:related}
\end{table}

\begin{table}[t]
\centering
\small
\def\arraystretch{1.0}  % height
\setlength\tabcolsep{0.5em}  % width
\scalebox{0.95}{
\begin{tabular}{l|c|c|c|c}
\toprule
Method          & Kosmos \cite{peng2023kosmos} & Ferret \cite{You2023FerretRA} & Shikra \cite{chen2023shikra} & Proposed \\ \midrule
Unified Arch.   & \xmark & \xmark & \cmark & \cmark  \\ 
Purely Textual  & \xmark & \xmark & \cmark & \cmark  \\ 
Pseudo Data     & \xmark & \xmark & \xmark & \cmark  \\ 
Video Domain    & \xmark & \xmark & \xmark & \cmark  \\ \bottomrule
\end{tabular}
}
\vspace{-0.5em}
\caption{Related Work Comparison: 
% We highlight key distinctions of our proposed method against three most similar related works. 
A unified architecture, purely textual inputs, pseudo data for scalable learning, and video domain operation distinguishes our proposed work from these prior methods.}
\label{tbl:related}
\end{table}
