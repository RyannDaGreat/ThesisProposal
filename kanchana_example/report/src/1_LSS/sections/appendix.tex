\section{Additional Details}

\begin{table}[t]
\centering\small
\caption[Additional Experiments with LSS]{
We report top-1 (\%) accuracy on the Kinetics-400 \cite{kinetics400} validation set for linear probing evaluation (left). All models are pre-trained on the training set of Kinetics-400 dataset. We also report a CLIP baseline for comparison purposes. Performance of our proposed approach is on-par with prior state-of-the-art and showcases improvements over our baseline method. We also report retrieval scores (top-right) for MSR-VTT and classification mAP (bottom-right) for Charades dataset.}
\vspace{0.5em}
\begin{minipage}{0.49\textwidth}
    \centering
    \setlength{\tabcolsep}{4pt}
    \scalebox{0.90}{
    \begin{tabular}{l|c|c}
    \toprule
    \rowcolor{Gray} 
    Method                                              & Backbone & Acc (\%) \\ \midrule
    CVRL \cite{qian2020spatiotemporal} \venue{(CVPR’21)}& R3D-101  & 67.6  \\
    BraVe \cite{recasens2021broaden} \venue{(ICCV'21)}  & R3D-50   & 66.7  \\ 
    Vi$^2$CLR \cite{Diba_2021_ICCV} \venue{(ICCV '21)}  & S3D      & 63.4  \\ 
    CORP \cite{Hu_2021_ICCV}        \venue{(ICCV '21)}  & R3D-50   & 66.6  \\ 
    SVT \cite{Ran2021SVT} \venue{(CVPR `22)}                              & ViT-B    & 68.1  \\ 
    VideoMAE \cite{Tong2022VidMAE} \venue{(NeurIPS `22)}                      & ViT-B    & 61.3  \\ 
    CLIP \cite{radford2021clip}                         & ViT-B    & 66.4  \\ \midrule
    LSS (ours)                                          & ViT-B    & 67.3  \\ 
    \bottomrule
    \end{tabular}}       
    \label{lss_tbl:sota_k400}
\end{minipage}
%
\begin{minipage}{0.49\textwidth}
    \centering
    \begin{tabular}{l|c|c|c} 
    \toprule \rowcolor{Gray} 
    Method & R@1  & R@5  & R@10 \\ \midrule
    CLIP \cite{radford2021clip}  & 30.6 & 54.4 & 64.3 \\
    LSS (ours)    & 33.8 & 58.2 & 70.3 \\ \bottomrule
    \end{tabular}
    
    \vspace{2em}
    
    \begin{tabular}{l|c}
    \toprule \rowcolor{Gray} 
    Method & Classification mAP \\ \midrule
    CLIP \cite{radford2021clip}  & 19.7         \\
    LSS (ours)  & 23.1         \\ \bottomrule
    \end{tabular}
\end{minipage}

\end{table}


\subsection{Prompting details}
\label{app:gpt_prompting}
Our proposed approach utilizes two sets of language based captions: categories and descriptions. While categories are obtained directly from the class labels of datasets (set of unique labels - e.g. 400 classes in Kinetics-400 dataset), the descriptions are generated automatically utilizing GPT-3 \cite{brown2020language}. For each category caption, we query GPT-3 to provide a set of descriptions and visual characteristics. 

In detail, we use the following two prompts to generate descriptions and visual characteristics:
\\
\texttt{prompt1 = "Give 4 different descriptions for the phrase: \{category\}?"} \\
\texttt{prompt2 = "List visual objects or characteristics usually seen with the action: \{category\}?"}
\\
The resulting two sets of captions are converted to text embeddings using our text-encoder, and a single average text embedding is computed. This averaged embedding is used as the description basis vector for that category. Also, the resulting dataset containing these category-description pairs is made available publicly. 


\subsection{Additional Experiments}

\noindent \textbf{Linear Probing Evaluation:}
We present more results for linear probing in \cref{lss_tbl:sota_k400} (left). Our proposed LSS improves over the baseline achieving competitive performance on Kinetics-400. 

\noindent \textbf{Text-to-video retrieval:}
An important characteristic of CLIP \cite{radford2021clip} is its retrieval ability across both language and visual modalities. In order to verify if proposed LSS retains these strengths, we run experiments on MSR-VTT text-to-video retrieval benchmark. We demonstrate how LSS improves over our baseline CLIP, reporting these results in \cref{lss_tbl:sota_k400} (top-right). 

\noindent \textbf{Charades Evaluation:}
 We explore an alternate task of zero-shot multi-label classification on the Charades video dataset. We report mAP results for this task in \cref{lss_tbl:sota_k400} (bottom-right) as an additional point of comparison.
