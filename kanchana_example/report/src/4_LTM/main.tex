\documentclass{article}

% \usepackage{corl_2025} % Use this for the initial submission.
% \usepackage[final]{corl_2025} % Uncomment for the camera-ready ``final'' version.
\usepackage[preprint]{corl_2025} % Uncomment for pre-prints (e.g., arxiv); This is like ``final'', but will remove the CORL footnote.

% Math commands - add as needed from individual papers
  % https://github.com/goodfeli/dlbook_notation.
%
% --- inline annotations
%
\newcommand{\red}[1]{{\color{red}#1}}
\newcommand{\todo}[1]{{\color{red}#1}}
\newcommand{\TODO}[1]{\textbf{\color{red}[TODO: #1]}}
% --- disable by uncommenting  
% \renewcommand{\TODO}[1]{}
% \renewcommand{\todo}[1]{#1}
\usepackage{amsthm}
\usepackage{graphicx}
\usepackage{animate}
\newcommand{\Var}{\textup{Var}}
\newcommand{\Cov}{\textup{Cov}}
\newcommand{\bE}{\mathbb{E}}
\newtheorem{proposition}{Proposition}
\newtheorem{example}{Example}


% \title{Language Conditioned Motion Representations \\for Robot Control}
% \title{Optical Flow as Universal Motion Representation \\for Robot Control}
\title{Pixel Motion as Universal Representation \\for Robot Control}

% NOTE: authors (\And and \AND commands) will be visible only in the camera-ready and preprint versions (i.e., when using the option 'final' or 'preprint'). For the initial submission the authors will be anonymized.


\author{
Kanchana Ranasinghe, 
Xiang Li, 
Cristina Mata, 
Jongwoo Park,
Michael S Ryoo \vspace{0.5em} \\
Stony Brook University \vspace{0.5em} \\
\texttt{kranasinghe@cs.stonybrook.edu} \\
}


\begin{document}
\maketitle

\vspace{-1.0em}
\begin{figure}[t]
    \centering
    \includegraphics[width=0.98\linewidth]{src/4_LTM/figures/teaser_fig.pdf}
    \caption[LangToMo Illustration]{
    Illustration of proposed dual-system VLA framework, LangToMo, with pixel motion representations.
    }
    \label{ltm_fig:teaser}
\end{figure}

%===============================================================================
% Abstracts should be a single paragraph, between 4--6 sentences long, ideally. 
\begin{abstract}
    We present \modelname, a vision-language-action framework structured as a dual-system architecture that uses pixel motion forecasts as intermediate representations. 
    Our high-level \textit{System 2}, an image diffusion model, generates text-conditioned pixel motion sequences from a single frame to guide robot control.
    Pixel motion—a universal, interpretable, and motion-centric representation—can be extracted from videos in a self-supervised manner, enabling diffusion model training on web-scale video-caption data.
    % an initial observation and action command.
    Treating generated pixel motion as learned \textit{universal representations}, our low level \textit{System 1} module translates these into robot actions via motion-to-action mapping functions, which can be either hand-crafted or learned with minimal supervision.
    System 2 operates as a high-level policy applied at sparse temporal intervals, while System 1 acts as a low-level policy at dense temporal intervals.
    This hierarchical decoupling enables flexible, scalable, and generalizable robot control under both unsupervised and supervised settings, bridging the gap between language, motion, and action.
    % All our code and models will be released publicly. 
    Checkout 
    % \href{https://anonymous.4open.science/w/LangToMo}{\texttt{anonymous.4open.science/w/LangToMo}}
    \href{https://kahnchana.github.io/LangToMo}{\texttt{kahnchana.github.io/LangToMo}}
    for visualizations.
\end{abstract}

% Two or three meaningful keywords should be added here
\keywords{Vision-Language-Action model, Self-Supervised, Diffusion} 
\vspace{0.5em}
%===============================================================================

\section{Introduction}
\label{sec:intro}

% Overview: motivate, what do we do 
Translating open-ended natural language instructions into robot actions is a cornerstone of flexible robot control. We identify two key requirements to enable this: (i) universal representations that support operating diverse embodiments \cite{nair2022r3m,Ren2025MotionTA,Zheng2025UniversalAF}, and (ii) benefiting from large-scale video-language data without action labels \cite{du2023learning,Gu2023SeerLI,Black2023ZeroShotRM,Ko2023LearningTA}. We explore their intersection, proposing \modelname, a vision–language–action framework structured as a \textit{dual-system architecture}, inspired by dual-process theories of cognition~\cite{Kahneman2011ThinkingFA} and recent hierarchical robotics frameworks~\cite{Belkhale2024RTHAH,Black20240AV,Shi2025HiRO,Nvidia2025GR00TNA,Intelligence2025pi05}. 
In our high level \textit{System 2} module, we use pixel motion as the robot action representation. We use image diffusion to learn to predict pixel motion from a single image (initial observation) conditioned on a language described action.
% Given how robot actions fundamentally correspond to some motion, LangToMo's high level \textit{System 2} module employs image-based diffusion models to map language to a universal motion representation, i.e. pixel motion structured as a 2D tensor, and is trained solely with video-language supervision.
Subsequently, our embodiment-specific low level \textit{System 1} deterministically projects these action representations into executable robot actions.

% Why pixel motion
We adopt pixel motion—the apparent motion of pixels between frames—as our \textit{universal motion representation}, because it is agnostic to embodiments, viewpoints, and tasks. 
By predicting pixel motion instead of full RGB images, \modelname captures essential motion patterns more efficiently than text-to-video generation \cite{du2023learning,Ko2023LearningTA,Gu2023SeerLI,Black2023ZeroShotRM}. 
Pixel motion can be freely computed from videos using self-supervised methods like RAFT~\cite{teed2020raft}, enabling scalable, weakly supervised training on large video-caption datasets, similar to prior work on predictive world models \cite{Gu2023SeerLI,Black2023ZeroShotRM}.

% Prior works
Optical flows, essentially a set of pixel motion (PM) between two consecutive frames, has been leveraged to enhance motion-focused video generation \cite{Liang2024MoVideoMV, Koroglu2024OnlyFlowOF}. 
\citet{Ko2023LearningTA} calculates flow from frame pairs to perform robot control, establishing the promise of this direction for robotics. In contrast, we directly generate PM from language and a single frame using our System-2 module, offering greater efficiency and performance. 
Our predicted PM serves as an interpretable intermediate representation for downstream systems (e.g., our System-1), enabling even unsupervised control via hand-crafted mappings.
% —unlike future-state features \cite{Hu2024VideoPP}, which are less interpretable \cite{Ko2023LearningTA}.
Alternate motion signals in image-space are used in works like \cite{Sudhakar2024ControllingTW,Shridhar2024GenerativeIA,Huang2024ReKepSR,Shi2025ZeroMimicDR}, but they rely on explicit dense annotations limiting training scalability, unlike our System-2 formulation. 

% Details of our setup
Sequences of PM generated by our System 2 are then be transformed into robot actions via \textit{System 1}, a fast and deterministic controller.
Specifically, System 1 consists of task-specific action mappings tailored to different embodiments and viewpoints. We explore two instantiations of System 1: (a) learning mappings directly from limited expert demonstrations, and (b) hand-crafting mappings by leveraging the interpretable nature of pixel motion (motivated by \cite{Ko2023LearningTA}). 
Connecting System 1 and System 2 forms our overall language-conditioned robot control framework, \modelname. This hierarchical formulation allows operating the expensive high-level System 2 at sparse temporal intervals while invoking the lightweight low-level System 1 at dense temporal intervals for efficient control.

In summary, our contributions are as follows:
\begin{itemize}[leftmargin=2em,noitemsep,topsep=0.0ex,itemsep=-1.0ex,partopsep=0ex,parsep=1ex]
    \item \textbf{Universal Action Representation:} pixel motion as a learnable, interpretable, and motion-focused representation for robot control tasks.
    \item \textbf{Simple \& Scalable Learning:} mapping natural language actions to motion representations (pixel motion sequences) with a conditional diffusion model trained on web-scale video-caption data, without requiring pixel-level or action trajectory annotations.
    \item \textbf{Robotics Application:} conversion of learned action representations into action policies with minimal supervision, enabling operation under zero-shot and even unsupervised settings.
\end{itemize}
We evaluate \modelname on both simulated and real-world environments, highlighting its effectiveness and generality across diverse robot control tasks.


%===============================================================================

\section{Related Work}
\label{sec:related_work}

\bhdr{Learning from Videos:} 
Robot learning has a rich history of leveraging videos to extract sub-goal information, learn strong representations, or build dynamics models for planning~\citep{lee2017learning,finn2017deep,sun2018neural,kurutach2018learning,pari2021surprising,nair2022r3m,shao2021concept2robot,chen2021learning,bahl2022human,sharma2019third,du2023learning,sivakumar2022robotic,Sudhakar2024ControllingTW,Ko2023LearningTA,Hu2024VideoPP,Ren2025MotionTA}.
Several recent works learn representations connected to language modality from video-caption data~\citep{du2023learning,Sudhakar2024ControllingTW,Ko2023LearningTA,Hu2024VideoPP}, but depend on additional action-trajectory annotations, pretrained segmentation models, or task-specific heuristics for robot control. 
We explore a similar direction, learning language-conditioned motion representations from video-caption data.  
In contrast to these works, our \modelname learns representations that are \textit{interpretable} and \textit{motion-focused}, which we use for robot control with no additional supervision. Our focus on pixel motion also allows faster learning of more generalizable representations.

\bhdr{Pixel Motion to Actions:} 
Robot navigation and control, especially in the context of aerial drones, has long benefited from optical flow representations \cite{Croon2021EnhancingOC,Lee2020AggressivePN,Hu2024SeeingTP,Argus2020FlowControlOF}, inspired by animal perception system that use optical flow for stable control and movement \cite{Gtz1968FlightCI,Arnold1974RHEOTROPISMIF,Baird2021TheEO,Ros2016OpticFS}. 
Video self-supervised learning has also extensively leveraged optical flow to learn motion representations \cite{Han2020SelfsupervisedCF,Sharma2022PixellevelCF}. 
In contrast to prior works, our \modelname is the first to model optical flow from a single image (pixel motion) conditioned on textual action descriptions, allowing language conditioned robot control. 


\bhdr{Diffusion-Based Motion Generation:}
Diffusion models have emerged as powerful generative frameworks capable of capturing complex data distributions through iterative denoising processes~\citep{ho2020denoising,ho2022video,ramesh2022hierarchical,zhang2023adding,singer2022makeavideo,villegas2022phenaki,ge2022long,kumari2023multiconcept,zhang2022motiondiffuse,ren2022diffusion,chen2023moddm,janner2022diffuser,du2023reduce,liu2022structdiffusion,wang2023diffusion,Chi2023DiffusionPV,Shridhar2024GenerativeIA}. 
While some works directly predict optical flow from image pairs~\citep{Saxena2023TheSE,Luo2024FlowDiffuserAO}, these tackle well-defined inputs. In contrast, \modelname generates pixel motion from a single image and language command, capturing the multimodal nature of future motions. By also conditioning on past motion, our approach introduces temporal grounding, making it well-suited for robot control.

\bhdr{Language-Conditioned Robotic Manipulation:}
Several recent works use vision-language models for robot control ~\citep{rt1,rt2,padalkar2023open,reed2022gato,wu2023unleashing,octo_2023,driess2023palm,kim2024openvla,yuan2024robopoint,niu2024llarva,zheng2024tracevla,Li2024LLaRASR,Zawalski2024RoboticCV,Hu2024VideoPP,Sudhakar2024ControllingTW,Ko2023LearningTA,Tian2024PredictiveID,Jeong2025ObjectCentricWM} taking advantage of large-scale training with web-scale vision-language data. In contrast to prior work using sequential language models, we learn motion representations under weak supervision (only video-caption data) using zero action trajectory annotations. We also utilize an image diffusion model similar to~\citep{Hu2024VideoPP,Sudhakar2024ControllingTW,Ko2023LearningTA} but differ by learning universal and interpretable motion representations directly, which even allows conversion to robot actions directly with no further training. 

% \kr{RT-H \cite{Belkhale2024RTHAH}, HiRobot \cite{Shi2025HiRO}, Pi0 \cite{Black20240AV}, Pi0.5 \cite{Intelligence2025pi05}, Groot-N1 \cite{Nvidia2025GR00TNA} has multi-system hierarchy, PIDM \cite{Tian2024PredictiveID}, Object Centric World Models \cite{Jeong2025ObjectCentricWM}, SuSIE \cite{Black2023ZeroShotRM},  Seer \cite{Gu2023SeerLI} uses predictive world model, VPT \cite{Baker2022VideoP} learns from unlabeled videos, UniAct \cite{Zheng2025UniversalAF} uses universal action representations in vector-quantized embedding space}


%===============================================================================

\begin{figure}[t]
    \centering
    \includegraphics[width=\linewidth]{src/4_LTM/figures/overview_fig.pdf}
    \caption[Overview of LangToMo]{
    \textbf{Overview of LangToMo:}
    (Left) We learn to forecast pixel motion as universal motion features from video-caption pairs using scalable, self-supervised training of a diffusion model.
    (Right) Our \textit{System 2} forecasts motion at sparse intervals ($k$), while \textit{System 1} maps it to dense action vectors at $j$ intervals ($j < k$).    
    }
    \label{ltm_fig:overview}
\end{figure}

\section{Methodology}
\label{sec:method}

We tackle the problem of robot control from natural language instructions by introducing a two-stage framework. Language and visual inputs are first encoded into pixel motion based representations, which are then decoded into robot actions. This dual-system architecture comprises: \textit{System 2}, a conditional image diffusion model that generates motion at sparse temporal intervals as a high-level controller; and \textit{System 1}, a task-specific low-level controller that maps these pixel motions to executable action vectors. An overview of our framework, \modelname, is shown in \Cref{fig:overview}.


\subsection{System 2: Pixel Motion Forecast}
\label{subsec:system2}

Optical flow estimation from frame pairs is a well-defined problem (exact solutions exist) that has been extensively studied \cite{xu2022gmflow,liu2019self,teed2020raft,Luo2024FlowDiffuserAO}. 
% In contrast, language conditioned OF estimation using a single frame involves a multi-modal output space: a single caption-frame pair could map to several distinct OF, since multiple optimal paths may exist to transform the current state towards achieving the language defined goal. 
In contrast, estimating pixel motion (PM) from a single image and language instruction is inherently multi-modal: a caption-frame pair may correspond to multiple valid flows, each representing a different trajectory toward the goal.
We use this challenging task as our self-supervision objective: learning a mapping from \textit{language to motion}. 
Furthermore, we incorporate temporal context by conditioning on the motion of a previous state.

% We chose a diffusion model architecture to learn this mapping function, $\gD$. This choice is motivated by the strong ability of diffusion models to represent such image-space relationships as demonstrated in numerous prior work \cite{ho2022video,ramesh2022hierarchical,singer2022makeavideo,villegas2022phenaki,zhang2022motiondiffuse}. 

Consider a video clip $\vx \in \sR^{t \times h \times w \times c}$ with $t, h, w, c$ for frames, height, width, and channels respectively. Also consider an embedding vector, $\vc$ representing the paired caption for that clip. Denoting the $i$-th frame of video as $\vx_i$, we define pixel motion, $\vy_{i, i+k}$, that corresponds to motion between frames $\vx_i \rightarrow \vx_{i+k}$ where $k$ is a constant. Our language to motion mapping function, $\gD$ becomes, 
% 
\begin{align}
    \hat{\vy}_{i, i+k} = \gD \left( \vx_i, \vy_{i-k,i}, \vc \ | \ \theta \right)
    \label{eq:mapping}
\end{align}
% 
where $\hat{\vy}_{i, i+k}$ is the predicted motion representation from the $i$-th state to $(i+k)$-th state \emph{without} knowing $\vx_{i+k}$. $\theta$ are learnable parameters.

% \bhdr{Learnable Mapping Function:}
We reiterate the multi-modal output aspect of our mapping described in \Cref{eq:mapping} (i.e. one to many mapping due to multiple optimal $\hat{\vy}_{i, i+k}$). Diffusion models have shown excellent abilities to model such distributions \cite{dhariwal2021diffusion,Chi2023DiffusionPV}. Considering the 2D structure present in our images and pixel motion, for $\mathcal{D}$ we elect to utilize a 2D conditional U-Net based diffusion model \cite{ramesh2022hierarchical} operating at pixel level. 
% (efficient inference through latent operation is left as future direction). 
% 
Our goal is to learn a set of parameters, $\theta$ for this diffusion model based mapping as, 
% 
\begin{align}
    \label{eq:train_obj}
    \argmin_{\theta} || \vy_{i, i+k} - 
     \gD \left( \vx_i, \vy_{i-k, i}, \vc \ | \ \theta \right) ||_2 
\end{align}
% 
that allows our language to motion mapping to perform instruction based robot control.  
Next we dive into the learning process of our diffusion based implementation for this mapping function. 

\subsection{Diffusion based Motion Representation Learning}
\label{subsec:ssl}

\bhdr{Background:} 
Diffusion Models generate data by progressively denoising corrupted signals, optionally conditioned on a goal input. While inference follows this iterative refinement process, training is conducted more efficiently using parallel denoising steps: the model is trained to predict less noisy versions of intermediate corrupted signals generated from clean data, a procedure analogous to teacher forcing (more details in \Cref{app:dm_detail}).


\bhdr{Architecture:}
The defacto architecture for diffusion based conditional image generation is the 2D conditional U-Net \cite{Ronneberger2015UNetCN}, which maps between 2D RGB images with an embedding based conditioning through cross-attention in the model intermediate layers. 
Basing off this setup, we modify the input and output heads to process 7 and 2 channel tensors respectively (instead of default 3 channel RGB). Two of the input channels and the two output channels correspond to our pixel motion target (noise input and clean output). 
The remaining 5 input channels correspond to our 2D-structured conditions:
previous pixel motion (2 channels) and current state image (3 channels).
These conditional inputs are not subject to the standard noise corruption schedule during training or inference (details in \Cref{app:dm_detail}). 
The textual embedding is provided as the default embedding condition. 
%  
Our channel modification to accommodate additional structured conditions allows a minimal design, retaining the general structure of the U-Net that is known to excel at 2D generative modeling.   
Such input channel concatenation based conditioning has been used in diffusion literature for different tasks \cite{Saxena2023TheSE,ho2022video} and is inspiration for our design.
% 
We illustrate this architecture in \Cref{fig:method} (left). 

\section{Language-based Self-Supervision (LSS)}
\label{lss_sec:method}
In this section, we present our proposal, Language-based Self-Supervision (LSS). The generality and robustness of shared image-language representation spaces such as that of CLIP \cite{radford2021clip} allow interesting manipulations of visual representations using language. We explore such manipulations under the setting of visual self-supervised learning focusing on video understanding. Self-supervised objectives can operate within a latent space constructed with language, retaining language alignment of learned visual representations. This allows better interpretability of representations as well as zero-shot inference. 
We discuss the four key components of our approach: backbone architecture, concept distillation objective, modifications to avoid collapse, and concept alignment objective.

\begin{figure}
\includegraphics[width=\textwidth]{src/1_LSS/figures/arch1.pdf}
\vspace{-1.5em}
\caption[Architecture Overview]{\small
\textbf{Architecture Overview:}
Our overall setup contains three components: visual teacher model (green), visual student model (red), and language model (blue). We utilize the text encoder of CLIP as our language model and extract \textit{concept vectors} relevant to action labels and descriptions of those actions. A visual encoder (containing a space-time backbone) is partially initialized with CLIP's visual encoder and used to obtain sample specific features. Generated concept vectors are used to project these features to a \textit{concept space} where our proposed \textit{concept distillation} and \textit{concept alignment} losses are applied.
}
\label{lss_fig:arch}
\end{figure}

\subsection{Backbone Architecture}
\label{lss_subsec:arch}
Our approach introduces a \textit{text classifier} to self-distillation based SSL works \cite{caron2021emerging, Ran2021SVT}, in place of the projector network.
% Our approach builds over self-distillation based SSL works \cite{caron2021emerging, Ran2021SVT} introducing a \textit{text classifier} in place of the projector network. 
Given a data sample $x$, let $x_1, x_2 \in \mathbb{R}^{(C,T,H,W)}$ be two augmented views generated using video specific transformations following \cite{Ran2021SVT}, where $C=3, T=8, H=W=224$ are channel, time, and spatial dimensions respectively. 

\textbf{Visual Encoder:}
A visual encoder, $\theta_v$, processes $x_i$ to produce feature $f_i \in \mathbb{R}^{768}$. We utilize the pre-trained image encoder of CLIP \cite{radford2021clip} expanded for temporal modelling using factorized space-time attention. The vision transformer variant of CLIP is selected to allow our factorized space-time attention. In particular, we use ViT-B/16 architecture for the the image encoder, in which for a given augmented view with $H=W=224$ and $T=8$, each transformer block processes 8 temporal and 196 spatial tokens separately in sequential order, and the embedding dimension of each token is $\mathbb{R}^{768}$. 
In addition to the input tokens from the data sample, one classification token \cite{devlin2018bert, dosovitskiy2020image} serves as the final feature vector output by the network, namely $f_i$, which is common to the CLIP image encoder. This classification token is inflated and processed suitably following \cite{bertasius2021timesformer} to accommodate our modifications for factorized space-time attention.  We follow \cite{bertasius2021timesformer} to zero-initialize additional time-attention parameters, achieving outputs identical to the pre-trained CLIP image encoder at start of training. 

\textbf{Text Classifier:}
Inspired by \cite{wu2022text4vis}, a set of $n$ language embeddings extracted from the CLIP text encoder, $\theta_t$, are used to construct the weight parameter of a linear layer (with no bias term), which we call our text classifier, $\theta_c$. The role of this text classifier is to project visual features $f_i$ to a vector space defined by those $n$ embeddings, producing $\tilde{f}_i \in \mathbb{R}^n$. Next we discuss these vector spaces (referred to as action concept spaces) and the text classifier module in detail.

\begin{figure}
\centering
\includegraphics[width=0.99\textwidth]{src/1_LSS/figures/concept_space1.pdf}
\vspace{-0.5em}
\caption[Concept Space Illustration]{\small
\textbf{Concept Spaces:}
We illustrate a toy concept space constructed with the three action concepts: run, swim, and walk. In this example, the text classifier projects visual feature $f_i$ into the 3-dimensional toy concept space to produce $\tilde{f}_i$. 
}
\label{lss_fig:cs}
\end{figure}

\subsection{Action Concept Spaces}
\label{subsec:concept_space}
Self-supervised learning approaches following exponential moving average (EMA) based self-distillation \cite{grill2020bootstrap,caron2021emerging,Ran2021SVT} utilize a projector network (MLP) to operate in a higher dimensional feature space. This is expected to minimize train-test domain gaps, handle noisy positive sample pairs, and better discriminate nuanced feature differences \cite{Balestriero2023ACO}. Focused on these notions, we propose an alternate \textit{concept space} composed of a set of basis vectors defined by language-based action concepts. Our language-based self-supervision objectives operate within such concept spaces.

\textbf{Concept Spaces:}
Building off the assumption that text encoder features capture subtle differences between distinct actions categories, we hypothesize that necessary nuanced distinctions between these actions will be better captured in our proposed concept spaces. The defining parameters of concept spaces are their basis vectors, $b_i$. Normalized embeddings (extracted from text encoder, $\theta_t$) of various natural language captions ($c_i$) relevant to action categories are used as these basic vectors. 
%
\begin{align}
    b_i &= {\theta_t(c_i)} \left. \right/ {||\theta_t(c_i)||^2_2} \\
    \mathbf{b} &= [b_1, b_2, ... \ b_n]^T \text{ ; } \mathbf{b} \in \mathbb{R}^{(n, d)} 
\end{align}
%
Note that these basis vectors are not necessarily orthogonal. As illustrated in \cref{lss_fig:cs}, a single set of basis vectors, $\mathbf{b}$, defines one action concept space.
We define two sets of basic vectors: action category vectors and action description vectors. Action category vectors relate to a single action label which is converted to a caption using textual prompting following \cite{radford2021clip}. Action description vectors are averaged embeddings of multiple descriptions and visual characteristics relevant to individual action categories. These two distinct sets of basic vectors lead to two distinct concept spaces which we name \textit{category concept space} and \textit{description concept space} respectively.  

\textbf{Category Concept Space:}
We explore 3 different strategies to construct the category concept space. The base setup uses action labels from Kinetics-400 \cite{kinetics400}, UCF-101 \cite{soomro2012ucf}, and HMDB-51 \cite{kuehne2011hmdb} datasets, leading to a set of 530 (400 + 101 + 51, ignoring overlaps) basis vectors. Our next goal of connecting LLMs and their action awareness occurs in the second two strategies. We utilize LLMs \cite{brown2020language} and visual-LLMs \cite{liu2023llava} to extract large sets of action category labels. While we explore this idea of expanding the basis vector set with LMM based additional action labels in \cref{sec:experiments}, the base setup containing a modest 530 categories was sufficient to improve downstream task performance.

\textbf{Description Concept Space:}
This space is constructed conditioned on the previous category concept space. For each action label used in the latter, we extract 4 distinct descriptions and a set of visual characteristics relevant to that action label using a large language model (LLM). The role of the LLM is to inject its world knowledge (i.e. awareness on videos, actions, and their attributes) into our learned representations during self-supervised learning. 
In detail, we prompt GPT-3 \cite{brown2020language} to generate such descriptions and characteristics using procedure outlined in \cref{app:gpt_prompting}. We highlight that GPT-3 is used here as an intelligent LLM containing world knowledge on videos and actions, in order to create natural language descriptions for given action category labels. 
The textual outputs generated for each action label are processed by our text encoder to produce multiple embeddings for a single action label. These embeddings are averaged to produce the corresponding basis vector for the description concept space. Note how this leads to a common dimensionality between the two concept spaces as well as one to one correspondences between the basic vectors of the spaces, which we leverage in our self-supervision objectives. 


\subsection{Concept Distillation}
\label{subsec:cd}
We now describe our primary self-supervised learning objective, concept distillation. Standard multi-view based self-supervision enforces a network to encode the common information between two augmented (distorted) views of a data sample \cite{Balestriero2023ACO}. This common information can be considered as the augmentation invariant signal present in the original data sample \cite{Balestriero2023ACO,Bardes2021VICRegVR}. In the case of self-distillation based approaches \cite{caron2021emerging,Ran2021SVT}, a higher dimensional feature space is utilized to enforce the self-supervision objectives. Instead, we propose to use action concept spaces as an alternative.
% , focusing on the case of video based self-supervision. 

Proposed concept distillation depends on an action concept space and visual video features aligned to the basis vectors of that space. Given our visual features $f_i \in \mathbb{R}^d$, we obtain projected $\tilde{f}_i \in \mathbb{R}^n$ as,  
%
\begin{align}
    % \mathbf{b} &= [b_1, b_2, ... \ b_n]^T \text{ ; } \mathbf{b} \in \mathbb{R}^{(n, d)} \notag \\
    \tilde{f}_i &= \mathbf{b} \ (\left. f_i \right/ ||f_i||^2_2) 
    = [b_1 \cdot f_i', b_2 \cdot f_i', ... \ b_n \cdot f_i']^T 
\end{align}
\textbf{Similarity Calculation: }
Projecting normalized visual video features to a concept space corresponds to calculating the dot-product similarity with each basic vector of the concept space. The projected vector $\tilde{f}_i$ can be viewed as a similarity \textit{score distribution} across all basis vectors of the concept space. Inspired by \cite{wu2022text4vis}, we implement this similarity calculation as a linear layer with weight matrix $\mathbf{b}$ and bias terms zero. We refer to this layer as the \textit{text classifier}. Similar to \cite{wu2022text4vis}, our text classifier remains frozen (no parameter updates), but in our case, this is to retain the original language distribution. 

\textbf{Concept Distillation Objective:}
Viewing projected features for two augmented views of a single video as score distributions, we argue that the underlying signal of the original video would relate to a unique score distribution to which score distributions of each view should be similar. Therein, following our EMA teacher based self-distillation setup (see \cref{lss_subsec:arch} for details), we enforce the score distribution to be consistent across views. 
Given two views $x_1, x_2$ of a single video, our teacher and student visual encoders process them respectively to produce $f_1, f_2$. The text classifier projects these to concept space, producing score distributions $\tilde{f}_1, \tilde{f}_2$. We obtain our objective, $\mathcal{L}_{\text{CD}}$ as:  
 %
 \begin{align}
    \label{eq:softmax}
    \hat{f}_i[k] &= \frac{\operatorname{exp}(\tilde{f}_i[k] / \lambda_i)}
                      {\sum_{j=1}^{n} \operatorname{exp}(\tilde{f}_i[j]/ \lambda_i) } \\
    \label{eq:weight}
    w_s &=  \operatorname{max}(\hat{f}_1) \\
    \label{eq:loss_cd}
    \mathcal{L}_{\text{CD}}(\tilde{f}_1, \tilde{f}_2) &= - w_s \cdot \sum_{j=1}^{n} \hat{f}_1[j] \operatorname{log} \hat{f}_2[j] 
 \end{align}
 %
The teacher and student score distributions, $\tilde{f}_1, \tilde{f}_2$, are softmax normalized in \cref{eq:softmax}, with temperature terms $\lambda_1=0.1, \lambda_2=1$ for sharpening only the teacher score distribution. A significance score $w_s$ is calculated for each sample in \cref{eq:weight}. In the softmax normalized teacher score distribution ($\hat{f}_1$), the maximum value is high when peaked at a single action concept and low when peaked at multiple action concepts. Considering the noisy nature of multi-peak teacher score distributions, we utilize $w_s$ to minimize their overall effect during training. Our overall $\mathcal{L}_{\text{CD}}$ is thus implemented as in \cref{eq:loss_cd}.   

\textbf{Distinct Concept Spaces:} Given the two distinct action concept spaces defined in \cref{subsec:concept_space}, we utilize two parallel text classifiers to implement each, and obtain two score distributions, one for each concept space. Defining score distributions $\tilde{f}_i^C, \tilde{f}_i^D$ for category and description concept spaces respectively, we apply our $\mathcal{L}_{\text{CD}}$ on each pair separately to obtain two losses $\mathcal{L}_{\text{CD}}^{\text{X}}$ for 
{\small X$\in$\{C,D\}} as:
%
\begin{align}
    \mathcal{L}_{\text{CD}}^{\text{X}} = \mathcal{L}_{\text{CD}}(\tilde{f}_1^{\text{X}}, \tilde{f}_2^{\text{X}})
\end{align}
%
We highlight how our concept spaces implemented as text classifiers are maintained intact by freezing the text classifier during training. This allows our approach to perform direct zero-shot inference, making concept distillation additionally advantageous over standard video SSL techniques. 


\subsection{Uniform Distribution Prior}
\label{subsec:udp}
Avoiding collapse is a key concern in SSL methods \cite{caron2021emerging,Ran2021SVT,Balestriero2023ACO} and recent self-distillation based approaches utilize feature sharpening and centering operations to avoid collapse \cite{caron2021emerging,Ran2021SVT}. While we similarly perform sharpening operations on the teacher outputs, given the nature of our action concept space, performing a learned vector mean subtraction based centering operations can break the meaningful structure of score distributions. Instead, we enforce a uniform distribution prior on the expected score distribution over the entire training dataset. The centering operation proposed in \cite{caron2021emerging} acts similarly pushing representations towards a uniform distribution while the sharpening operation counters its effect. We approximate expectation over the dataset as a moving average of mean score distributions at each train iteration and the uniform prior is enforced as: 
%
\begin{align}
    \hat{f}_{\text{MA}}^{\text{X}} &= \tau \cdot \hat{f}_2^{\text{X}} + (1 - \tau) \cdot \hat{f}_{\text{MA}}^{\text{X}} \\
    \label{eq:up}
    \mathcal{L}_{\text{UP}}^{\text{X}} &= - \frac{1}{n} \sum_j \operatorname{log} \hat{f}_{\text{MA}}^{\text{X}}[j]
\end{align}
%
where the hyper-parameter $\tau=0.5$ is fixed during training. We highlight that $\mathcal{L}_{\text{UP}}$ is necessary for convergence with concept distillation and is added to the concept distillation objective, $\mathcal{L}_{\text{CD}}^{\text{X}}$. 

%\begin{alignat}{2}
%	x + y &= 5  &\quad&\text{first equation} \\
%	2x - y &= 1  &&\text{second equation}
%\end{alignat}


\subsection{Concept Alignment}
Aligning action category labels and their descriptions or attributes within some embedding space has been explored in video SSL under multiple settings \cite{chen2021erzsar,zhu2018ur}. Motivated by these promising results, we explore how such alignment can be integrated to improve our framework with \emph{concept spaces}. In \cref{subsec:concept_space}, we define two distinct action concept spaces constructed from category labels and detailed category descriptions respectively. We hypothesize that explicit alignment of video features between these two spaces based on their one to one relationship can learn additional information. Therein, we introduce our concept alignment objective, $\mathcal{L}_{\text{CA}}$, as follows:
%
\begin{align}
    \mathcal{L}_{\text{CA}} = \mathcal{L}_{\text{CD}}(\tilde{f}_1^{\text{C}}, \tilde{f}_2^{\text{D}}) + \mathcal{L}_{\text{CD}}(\tilde{f}_1^{\text{D}}, \tilde{f}_2^{\text{C}})
\end{align}
%
\textbf{Overall SSL Objective:}
Reusing $\mathcal{L}_{\text{CD}}$ from \cref{eq:loss_cd}, we match score distributions across our two concept spaces instead of within a single concept space. $\mathcal{L}_{\text{CD}}(\tilde{f}_1^{\text{C}}, \tilde{f}_2^{\text{D}})$ aligns student description score distribution $\tilde{f}_2^{\text{D}}$ to teacher category score distribution $\tilde{f}_1^{\text{C}}$ while $\mathcal{L}_{\text{CD}}(\tilde{f}_1^{\text{C}}, \tilde{f}_2^{\text{D}})$ aligns student category score distribution $\tilde{f}_2^{\text{C}}$ to teacher description score distribution $\tilde{f}_1^{\text{D}}$. Combining all terms, we obtain:
% This leads to our overall self-supervised training objective:
%
\begin{align}
\label{eq:overall}
\mathcal{L} = (\mathcal{L}_{\text{CD}}^\text{C} + \mathcal{L}_{\text{UP}}^\text{C}) + (\mathcal{L}_{\text{CD}}^\text{D} + \mathcal{L}_{\text{UP}}^\text{D}) + \mathcal{L}_{\text{CA}}
\end{align}


\subsection{Concept Space Variants}
Our baseline concept space (described in \cref{subsec:concept_space}) utilizes labels from three standard video datasets (Kinetics-400, UCF-101, HMDB-51). However, we want to ensure scalability with more data and no label leakage to downstream evaluation tasks. With this goal, we propose 2 additional variants of action concept spaces tagged LSS-B and LSS-C. These variants do not use any form of ground truth textual labels from datasets. Moreover, they leverage the world awareness (i.e. knowledge on videos and actions) of LLMs to generate extensive action categories. Our baseline setup is hereafter referred as LSS-A.  

For LSS-B, we use GPT-3 \cite{brown2020language} to generate a large set of action labels. We first prompt GPT to categorize all common human actions / activities into 20 groups. For each group, we again ask GPT to generate at least 100 visually diverse action categories. These are all collected to create a set of 2000 action labels. We then use projections of these labels in CLIP text-encoder representation space to eliminate labels of high semantic similarity (spectral clustering in feature space from \cite{ranasinghe2022perceptual} to identify similar features), achieving 1000 diverse action categories. So our 1000 action categories for LSS-B are generic, not tied to any of our training datasets, and scalable with more data.

For LSS-C, we generate a label set using only videos from the training dataset. We use PCA based clustering to identify 2000 representative videos from a randomly sampled subset (50,000) of our training dataset and then use image-captioning models (LLaVa \cite{liu2023llava}) on video center frames to generate a diverse set of 2000 action labels. This is further reduced to 500 eliminating labels that are similar in feature space of the CLIP text encoder. In this case, our generated labels are tied to the training dataset, but uses no textually annotated category labels. We use only the videos (and an image-to-text captioning model) to generate our label set, still resulting in a scalable framework.

Note that each of these alternate strategies relates to construction of our category concept space. Given the selected set of textual category labels of this space, the description concept space is constructed in the same common way (as described in \cref{subsec:concept_space}).  We also reiterate that LSS-B and LSS-C variants use no category information from train / test datasets. 



\bhdr{Calculating Pixel Motion Ground-truth:}
% 
We utilize the RAFT algorithm \cite{teed2020raft} to calculate our target pixel motion $\vy_{i, i+k}$, using frames $\vx_i$ and $\vx_{i+k}$. 
This is an efficient iterative algorithm that calculates a good estimate of optical flow, in other words, pixel motion. Each pixel motion, $\vy_{i, i+k} \in \sR^{h \times w \times 2}$, contains two channels for spatial directions, that are normalized to a $(0, 1)$ range. All motion is represented within this 2D space - extensions to a third depth dimension are left as a future direction. Our experiments indicate the sufficiency of such 2D spaces to encode motions relevant to robot actions. 
We note that given the presence of background motions in both natural and simulation images (e.g. shadows moving with objects), this target pixel motion contains noise that is not directly relevant to the underlying motion, underscoring the challenging nature of our self-supervision objective.    

\bhdr{Previous Pixel Motion Representation:}
The other input signal to our mapping function is past frames pixel motion. Motivated by success of teacher forcing in generative modeling of both language \cite{Radford2019LanguageMA} and videos \cite{Song2025HistoryGuidedVD}, we use the target pixel motion of previous time steps during our System-2 training. 
% 
We also note the importance of representing pixel motion relative to current state as our mapping function is conditioned on the current image (details in \Cref{app:relative_of}). Similar findings are observed in image-pair based optical flow calculation literature \cite{Saxena2023TheSE}.  

\bhdr{Language Instruction Embeddding:}
The primary input conditioning of our mapping function is the natural language based action description that is used to control the generated motions. Following prior robotics literature \cite{padalkar2023open}, we use a Universal Sentence Encoder model \cite{Cer2018UniversalSE} to convert textual instructions to fixed size embedding vectors. This embedding model is trained to capture sentence level meanings. We use an off-the-shelf pretrained version, keeping all model parameters unchanged (more details in \Cref{app:lang_embed}).   


\bhdr{Training:}
Our training uses the standard diffusion denoising objective \cite{ho2020denoising} between predicted ($\hat{\vy}_{i, i+k}$) and target ($\vy_{i, i+k}$) pixel motion. The conditional 2D inputs, $\vx_i$ and $\vy_{i-k, i}$ are not subject to a noising schedule. The image condition, $\vx_i$, remain uncorrupted while the previous pixel motion, $\vy_{i-1,i}$, is set to random noise or a partially corrupted version to align with inference settings. We also introduce zero motion to ends of videos such that when textual instruction is complete, those visual states map to zero motion. More details in \Cref{app:dm_detail}. 
%

\bhdr{Inference:}
We forecast pixel motion from $i$ to $i+k$ timestamp using a 25-step DDIM schedule with only the current image observation $\vx_i$. At the initial step, the model only takes the image $\vx_i$ (state observation), language instruction $c$, and random noise as the previous pixel motion. For subsequent steps, the previously predicted motion is reused, enabling sequential pixel motion generation that drives the system toward fulfilling the language command.


\subsection{System 1: Pixel Motion to Action Mapping}
\label{subsec:system1}

Our System 2 produces pixel motion conditioned on a given state-instruction pair.
We next detail how these pixel motion representations are mapped into action vectors that directly control the robot. 
Consider a mapping function, $\gF$, operating at dense temporal intervals:
%
\begin{align}
    \label{eq:actions}
    \hat{\va}_{i+j} = \gF \left( \hat{\vy}_{i, i+k}, \vx_i, \vx_{i+j} \right),
\end{align}
%
where $ j \in \left[0,k\right]$, $i$ is a multiple of $k$ (for a hyperparameter $k$), and $\hat{\va}_{i+j}$ denotes the predicted action vector for the $(i+j)$-th state. 
An overview of this formulation is shown in \Cref{fig:overview} (right).

While \textit{System 2} is trained as a general-purpose motion generator across diverse embodiments, viewpoints, and environments, 
action vectors $\va_i$ are inherently embodiment-specific.
Hence, we design \textit{task-specific} mapping functions to serve as \textit{System 1 (Action Mapping)}, converting pixel motion into executable robot actions.

\bhdr{Learned Mapping:}
We implement a neural network-based mapping function that can be trained using ground-truth action trajectories. 
Given the 2D spatial structure of the inputs to $\gF$ (i.e., $\hat{\vy}_{i,i+j}$, $\vx_i$, $\vx_{i+j}$), we channel-concatenate them and feed the resulting tensor to a lightweight vision transformer to predict action vectors.
This architecture is illustrated in \Cref{fig:method} (right).
The network is trained on a limited amount of task-specific demonstration data. 
Connecting this learned \textit{System 1} with \textit{System 2} following \Cref{eq:actions}, we obtain a complete pipeline for language-conditioned robot control.
We refer to the resulting system, which uses a supervised learned mapping, as \modelshort-S.

\bhdr{Hand-Crafted Mapping:}
The interpretable nature of pixel motion also enables hand-crafted designs for $\gF$.
We refer to the resulting pipeline based on hand-crafted mappings as \modelshort-H.
%
For simulated environments where ground-truth segmentations and depth maps are available, we follow the methodology in~\cite{Ko2023LearningTA} to define action mappings, ensuring a fair evaluation of the utility of our pixel motion predictions compared to prior works.
%
For real-world robot control, we construct viewpoint-specific hand-crafted mappings following~\cite{Li2024LLaRASR}.
Further details on both learned and hand-crafted mappings are provided in \Cref{app:handcraft_map}.

We highlight how our System 1 operates at a frequency different to our System 2, allowing a balance between efficiency and dense control. Our System 1 is also designed to be lightweight, given how it performs an almost deterministic mapping.  

% \bhdr{MPC based Mapping}
% We demonstrate the applicability of \modelname in robotic manipulation tasks. By integrating the framework with a motion-based model predictive control (MPC) algorithm, we enable robots to perform actions described by natural language commands without prior task-specific training, as explored in recent studies \cite{bharadhwaj2023zero}.


%===============================================================================

\section{Experimental Results}
\label{sec:result}

We conduct experiments on 15 tasks spanning both simulated and real-world environments to highlight the strong performance of our proposed \modelname framework. 
We also present multiple ablations to justify key design choices within our method.

\bhdr{Implementation Details:}
Our framework consists of \textit{System 2 (Motion Generation)} containing a diffusion model, and \textit{System 1 (Action Mapping)} containing either a learned or hand-crafted mapping function.
%
We pretrain the diffusion model on a subset of the OpenX dataset~\cite{padalkar2023open}, followed by optional fine-tuning on downstream task datasets. 
Pretraining is performed for 300,000 iterations with a learning rate of $1\text{e-}4$, following a cosine learning rate schedule with 500 warmup steps, using 8 A100 GPUs (48GB) with a per-device batch size of 32 samples. 
Fine-tuning is performed for 100,000 iterations on 4 A5000 GPUs (24GB) with a batch size of 32 and a learning rate of $1\text{e-}5$, again following a cosine schedule with 500 warmup steps.
%
The learned action mapping (System 1) is trained separately using a vision transformer for 10,000 iterations on a single A5000 GPU with a batch size of 128 and a learning rate of $1\text{e-}4$.
% 
During inference of our System 2 diffusion model, we use a DDIM scheduler with 25 steps to generate flow sequences, starting from noise. For each invocation of System 2, we run System 1 for 10 control steps (or until convergence in the hand-crafted setting). This hierarchical procedure is repeated until the episode terminates.


\subsection{MetaWorld Simulated Environment}

MetaWorld~\cite{yu2019meta} is a simulated benchmark containing several robot manipulation tasks with accompanying natural language instructions. 
Each task episode corresponds to successfully completing an action described in language. 
The environment utilizes a Sawyer robot arm.

\bhdr{Training:}
We train \textit{System 2} (diffusion model) first on the OpenX subset, followed by additional training on 165 MetaWorld videos (identical to the split used in~\cite{Ko2023LearningTA}). 
For the learned variant of \textit{System 1}, we train on 20 expert demonstrations per task.
We also implement a hand-crafted variant of System 1, following the design in~\cite{Ko2023LearningTA} to ensure fair comparison.

\bhdr{Evaluation:}
Following evaluation settings identical to~\cite{Ko2023LearningTA}, we evaluate each policy across 11 tasks.
For each task, videos are rendered from 3 distinct camera poses, with 25 randomized trials (different initial positions of the robot arm and objects) for each view.
We replicate multiple baselines from~\cite{du2023learning,Ko2023LearningTA} under common settings for comparison.

\bhdr{Results:}
We present the success rates for the 11 tasks and the average across tasks in \Cref{table:res_mw}.
Both our \modelshort-H and \modelshort-S variants achieve strong overall performance, highlighting the effectiveness of our framework.
%
Notably, several strong approaches~\cite{du2023learning,Ko2023LearningTA} exhibit moderate success rates, underscoring the difficulty of the benchmark.
An important point of comparison is the AVDC (flow) baseline from~\cite{Ko2023LearningTA}, which also uses pixel motion prediction but differs in model architecture, flow representation, and training procedures.
The improved performance of \modelname over AVDC demonstrates the impact of our design choices.

\begin{table}[t]
\centering
\small
\def\arraystretch{1.2}  % height
\setlength\tabcolsep{0.6em}  % width
\scalebox{0.80}{
\begin{tabular}{lcccccccccccc}
\toprule
& \rotatebox{75}{door-open} & \rotatebox{75}{door-close} & \rotatebox{75}{basketball} & \rotatebox{75}{shelf-place} 
& \rotatebox{75}{btn-press} & \rotatebox{75}{btn-top} & \rotatebox{75}{faucet-close} & \rotatebox{75}{faucet-open} 
& \rotatebox{75}{handle-press} & \rotatebox{75}{hammer} & \rotatebox{75}{assembly} & \rotatebox{75}{Overall} \\
\midrule
BC-Scratch     & 21.3 & 36.0 & 0.0 & 0.0 & 34.7 & 12.0 & 18.7 & 17.3 & 37.3 & 0.0 & 1.3 & 16.2 \\
BC-R3M         & 1.3 & 58.7 & 0.0 & 0.0 & 36.0 & 4.0 & 18.7 & 22.7 & 28.0 & 0.0 & 0.0 & 15.4 \\
UniPi (With Replan) & 0.0 & 36.0 & 0.0 & 0.0 & 6.7 & 0.0  & 4.0 & 9.3 & 13.3 & 4.0 & 0.0 & 6.1 \\
AVDC (Flow)    & 0.0 & 0.0 & 0.0 & 0.0 & 1.3 & 40.0 & 42.7 & 0.0 & 66.7 & 0.0 & 0.0 & 13.7 \\
AVDC (Default) & 72.0 & 89.3 & 37.3 & \textbf{18.7} & 60.0 & 24.0 & 53.3 & 24.0 & 81.3 & \textbf{8.0} & 6.7 & 43.1 \\ \rowcolor{Gray}
LTM-H (Ours)    & 76.0 & 94.7 & 38.0 & 15.2 & \textbf{82.0} & \textbf{84.7} & 41.3 & 33.3 & 97.3 & 4.2 & \textbf{6.9} & 52.1 \\ \rowcolor{Gray}
LTM-S (Ours)    & \textbf{77.3} & \textbf{95.0} & \textbf{39.0} & \textbf{18.7} & \textbf{82.0} & 84.3 & \textbf{46.7} & \textbf{35.3} & \textbf{98.0} & 6.7 & \textbf{6.9} & \textbf{53.6} \\
\bottomrule
\end{tabular}
}
\vspace{0.5em}
\caption[Results on MetaWorld Environment]{
\textbf{Results on MetaWorld Environment:}
We report the mean success rate across tasks. Each entry of the table shows the average success rate aggregated from $3$ camera poses with $25$ seeds for each camera pose.}
\label{ltm_table:res_mw}
\end{table}
% \input{tables/results_libero}
\begin{table}[t]
\centering
\small 
\begin{minipage}{0.48\textwidth}
    \def\arraystretch{1.3}  % height
    \setlength\tabcolsep{0.4em}  % width
    \scalebox{0.75}{
    \begin{tabular}{lcccccc}
    \toprule
    Method  &  \makecell{Video Only\\ Training} & T1 & T2 & T3 & T4 & Avg \\ \midrule
    RT-2 Style \cite{rt2}            & \xmark &   0 &  0 & 0  & 0  &  0   \\
    LLaRA \cite{Li2024LLaRASR} & \xmark &  70 & 80 & 55 & 55 & 65.0 \\ 
    AVDC \cite{Ko2023LearningTA}& \cmark & 10 & 20 & 0 & 0  & 15.0 \\ \rowcolor{Gray} 
    LTM-H (ours)  & \cmark &  \textbf{80} & \textbf{70} & \textbf{65} & \textbf{60} & \textbf{68.8} \\ \bottomrule
    \end{tabular}}
    \vspace{0.5em}
    \caption[Real World Task Performance]{
    \textbf{Real World Task Performance:} 
    We follow the setup in LLaRA \cite{Li2024LLaRASR} to evaluate model performance on real world tasks under fine-tuned settings.
    % No Traj. Free Training refers to using no action trajectories for training the model (learning directly from videos). 
    }
    \label{ltm_table:real}
\end{minipage}
% 
\hspace{0.02\textwidth}
% 
\begin{minipage}{0.48\textwidth}
    \def\arraystretch{1.3}  % height
    \setlength\tabcolsep{0.5em}  % width
    \scalebox{0.85}{
    \begin{tabular}{lcccccc}
    \toprule
    Method  & \makecell{Video Only\\ Training} & T1 & T2 & T3 & T4 & Avg \\ \midrule
    RT-2 Style \cite{rt2}        & \xmark &   0 &  0 & 0  & 0  &  0   \\
    LLaRA \cite{Li2024LLaRASR}        & \xmark &  40 & 20 & 10 & 20 & 22.5 \\
    AVDC \cite{Ko2023LearningTA} & \cmark &  0 &  0 & 0  & 0  &  0 \\
    GPT-4o \cite{Achiam2023GPT4TR} & \cmark &  20 & 30 & 10 & 15 & 18.8 \\ \rowcolor{Gray} 
    LTM-H (ours) & \cmark &  40 & 30 & 35 & 30 & 33.8 \\ \bottomrule
    \end{tabular}}
    \vspace{0.5em}
    \caption[Zero-Shot Transfer on Real World Tasks]{
    \textbf{Zero-Shot Transfer on Real World Tasks:} Evaluations follow settings in \cite{Li2024LLaRASR}.
    }
    \label{ltm_table:real_zs}
\end{minipage}
\vspace{1em}
\end{table}


% \begin{table}[t]
% \centering
% \def\arraystretch{1.3}  % height
% \setlength\tabcolsep{0.8em}  % width
% \scalebox{0.85}{
% \begin{tabular}{lccccccc}
% \toprule
% Method       &Zero-Shot&  \makecell{Action GT\\Free Training}   & {Task-1} & {Task-2} & {Task-3} & {Task-4} & {Average} \\ \midrule
% RT-2         & \cmark  & \xmark &   0 &  0 & 0  & 0  &  0   \\
% RT-2         & \xmark  & \xmark &   0 &  0 & 0  & 0  &  0   \\
% LLaRA        & \cmark  & \xmark &  40 & 20 & 10 & 20 & 22.5 \\
% LLaRA        & \xmark  & \xmark &  90 & 85 & 75 & 80 & 82.5 \\ 
% % LLaRA        & \xmark  & \xmark &  70 & 80 & 55 & 55 & 65.0 \\ 
% \midrule
% AVDC         & \cmark  & \cmark &   0 &  0 & 0  & 0  &  0 \\
% AVDC         & \xmark  & \cmark &  10 & 20 & 0  & 0  & 15.0 \\ 
% GPT-4o       & \cmark  & \cmark &  20 & 30 & 10 & 15 & 18.8 \\ \rowcolor{Gray} 
% LTM-H (ours) & \cmark  & \cmark &  40 & 30 & 35 & 30 & 33.8 \\ \rowcolor{Gray} 
% LTM-H (ours) & \xmark  & \cmark &  \textbf{80} & \textbf{70} & \textbf{65} & \textbf{60} & \textbf{68.8} \\ \bottomrule
% \end{tabular}}
% \vspace{0.5em}
% \caption{
%     \textbf{Results on Real World Tasks:} 
%     We follow the setup in LLaRA to evaluate model performance on real world tasks under zero-shot and fine-tuned settings.
%     No Traj. Free Training refers to using no action trajectories for training the model (learning directly from videos). 
% }
% \label{table:real}
% \vspace{-1em}
% \end{table}

\subsection{Real-World Environment}

We next evaluate on 4 challenging tasks in an xArm Table Top environment, constructed following the real-world setup in~\cite{Li2024LLaRASR}. 
Examples of these tasks are shown in \Cref{fig:task_vis}.
The tasks involve tabletop manipulations specified by language commands (details in \Cref{app:real_world}).

\bhdr{Training:}
We train \textit{System 2} (diffusion model) on the OpenX subset, followed by optional fine-tuning on 10 videos per task collected in the same real-world environment.
We replicate the AVDC baseline~\cite{Ko2023LearningTA} by training under identical conditions. 
All other baselines are implemented following the settings used in~\cite{Li2024LLaRASR}.
For \textit{System 1}, we construct a hand-crafted mapping function based on~\cite{Ko2023LearningTA,Li2024LLaRASR} (details in \Cref{app:real_world}).

\bhdr{Evaluation:}
We follow evaluation settings identical to~\cite{Li2024LLaRASR}, evaluating each policy across 4 tasks with a fixed camera view and 20 randomized trials per task.
Each trial uses different initial positions of the objects present in the environment.

\bhdr{Results:}
We present results in \Cref{table:real,table:real_zs} to highlight the strong performance of \modelname (baseline details in \Cref{app:baselines}). The difficulty of these tasks is apparent by the moderate results from recent methods like LLaRA \cite{Li2024LLaRASR}.
% Zero-shot evaluation corresponds to training only on the OpenX subset, while fine-tuning involves training with a small set of expert demonstrations from the target environment.
Notably, despite relying on heuristic-based hand-crafted mappings in \textit{System 1}, \modelname outperforms several state-of-the-art baselines such as RT-2~\cite{rt2} and LLaRA~\cite{Li2024LLaRASR}, all without requiring action trajectory labels during training.
Our framework learns directly from videos paired with natural language captions, showing the promise of this direction.

\begin{figure*}[t]
    \centering
    \begin{minipage}{\linewidth}
    {\small
        \hspace{0.6em} {\texttt{Start State}} 
        \hspace{0.4em} {\texttt{Predicted Motion Overlaid on Intermediate States}}
        \hspace{0.4em} {\texttt{End State}}
    }
    \end{minipage}
    \begin{minipage}{0.03\linewidth}
    \centering
    \begin{tabular}{c}
         \rotatebox{90}{\texttt{Task-1}}   \\[2.7em] 
         \rotatebox{90}{\texttt{Task-2}}   \\[2.8em] 
         \rotatebox{90}{\texttt{Task-3}}   \\[2.9em] 
         \rotatebox{90}{\texttt{Task-4}}  \\
    \end{tabular}
    \end{minipage}
    \begin{minipage}{0.96\linewidth}
    % 
    \begin{minipage}{\linewidth}
        \centering\centering
        \includegraphics[width=\linewidth]{src/4_LTM/figures/task_vis/task1.png}
    \end{minipage}
    \begin{minipage}{\linewidth}
        \centering\centering
        \includegraphics[width=\linewidth]{src/4_LTM/figures/task_vis/task2.png}
    \end{minipage}
    \begin{minipage}{\linewidth}
        \centering\centering
        \includegraphics[width=\linewidth]{src/4_LTM/figures/task_vis/task3.png}
    \end{minipage}
    \begin{minipage}{\linewidth}
        \centering\centering
        \includegraphics[width=\linewidth]{src/4_LTM/figures/task_vis/task4.png}
    \end{minipage}
    % 
    \end{minipage}
    \caption[Real World Tasks]{
    \textbf{Real World Tasks:} 
    We illustrate the four real-world tasks following LLaRA~\cite{Li2024LLaRASR}. Start and end states are shown in the first and last columns, with predicted pixel motion (color indicates motion direction) overlaid on intermediate states. 
    LangToMo performs these challenging tasks successfully (see results in \Cref{ltm_table:real}).
    }
    \label{ltm_fig:task_vis}
\end{figure*}

\begin{table}[t]
\centering
\small
\begin{minipage}{0.38\textwidth}
\vspace{-0.5em}
\caption[LTM Ablation Study]{
    \textbf{Ablation Study:} 
    We report mean success rate (overall) on MetaWorld benchmark with our LTM-S variant. (left) Results highlight importance of key components in our System-2 model. 
    (right) Results justify several high-level design choices of our framework.
}
\label{ltm_table:ablate}
\end{minipage}
% 
\hspace{0.01\textwidth}
% 
\begin{minipage}{0.30\textwidth}
    \def\arraystretch{1.3}  % height
    \setlength\tabcolsep{0.6em}  % width
    \scalebox{0.75}{
    \begin{tabular}{ccccc}
    \toprule
    Img  & Lang & Prev Flow & PT & Overall \\ \midrule \rowcolor{Gray}
    \cmark & \cmark & \cmark & \cmark & 53.6 \\
    \cmark & \cmark & \cmark & \xmark & 53.1 \\
    \cmark & \cmark & \xmark & \xmark & 50.2 \\
    \cmark & \xmark & \xmark & \xmark & 39.7 \\
    \xmark & \cmark & \xmark & \xmark & 15.4 \\
    \bottomrule
    \end{tabular}}    
\end{minipage}
%
\hspace{0.02\textwidth}
% 
\begin{minipage}{0.26\textwidth}
    \def\arraystretch{1.3}  % height
    \setlength\tabcolsep{0.6em}  % width
    \scalebox{0.75}{
    \begin{tabular}{lc}
    \toprule
    Method  & Overall \\ \midrule \rowcolor{Gray}
    Ours (default)              & 53.6 \\
    No diffusion                & 16.2 \\
    CA instead of concat        & 15.8 \\
    Sys-1 \& 2 same freq       & 48.7 \\
    Only learned Sys-1          & 15.8 \\
    \bottomrule
    \end{tabular}}     
\end{minipage}
% 
\end{table}

\subsection{Ablation Studies}
\label{subsec:ablation}

We conduct a series of ablative studies with LTM-S on the MetaWorld benchmark to evaluate the importance of key components within \modelname. Results are summarized in \Cref{table:ablate}.

\bhdr{System 2 Input Conditioning \& Pretraining:}
Removing visual (``Img"),  language (``Lang"), or previous flow (``Prev Flow") conditional inputs to the diffusion model significantly reduces performance, highlighting importance of each conditioning signal.
On the other hand, removing diffusion model pretraining (``PT'') leads to a modest performance drop, indicating that while pretraining aids convergence and performance, the framework remains effective with limited finetuning alone.

\bhdr{Simpler Baselines:}
Replacing diffusion (``No diffusion") with an autoencoder breaks System-2 learning process. 
Modifying conditioning strategy to cross-attention (``CA instead of concat") also degrades performance. 
% We attribute this to loss of spatial information when performing cross-attention with a spatially-averaged visual embeddings. 
Skipping the iterative System-1 design (running System-1 at same frequency), and generating multiple actions per System-2 generated motion at once (``Sys-1 \& 2 same freq") also degrades success rates, validating our design choices.
Additionally, bypassing intermediate motion representations (``Only learned Sys-1") leads to poor results, underscoring the necessity of our two-stage architecture. See \Cref{app:ablate} for a detailed discussion. 


%===============================================================================

\section{Conclusion}
\label{sec:conclusion}

We presented \modelname, a scalable vision-language-action framework that decouples motion generation and action execution through a dual-system architecture. 
By leveraging diffusion models to learn universal pixel motion representations from video-caption data, our \textit{System 2} enables generalizable, interpretable motion planning without dense supervision.
These motions are translated into robot actions by \textit{System 1}, using either learned or hand-crafted mappings tailored to specific embodiments.
Extensive experiments across simulated and real-world environments demonstrate strong performance of \modelname, highlighting the promise of universal motion representations as a bridge between language, vision, and action for scalable robot learning.


\subsection*{Limitations}

LangToMo is pretrained on large-scale video-caption data, but relies on hand-crafted or learned action mappings in System 1 which can be costly for each new downstream task. Learning robust, transferable mappings remains an open challenge.
% 
Also, our framework models motion using 2D pixel motions, which currently lacks depth cues. Extending to 3D motion representations is left as a future direction.
% 
In terms of speed, despite operating at sparse intervals, System 2 relies on diffusion models that remain computationally expensive at inference time, limiting use in resource-constrained deployments. This is another future direction we hope to explore further. 
% 
Finally, we currently do not account for ego motion in training videos: we limit our training to fixed camera videos (no ego motion). A key next direction is extending our System-2 training to include videos with ego motion, which would allow scaling to any kind of video.

%===============================================================================

% \clearpage
% The contributions are automatically included only in the final and preprint versions of the paper.
\contributions{
KR led the project formulating the initial idea, coding the implementation, and performing most experiments. 
XL proposed several design choices of the approach, built the initial setup for real world experiments, and discussed all aspects of the project. 
CM contributed to ideas on experiment design, performed several real world experiments, and discussed most aspects of the project. 
JP helped setup human and robot demonstrations in real world, supported data collection and evaluations, and discussed most aspects of the project. 
MR organized the project, set the research direction, and discussed all aspects of the project idea, scope, and implementation.
}

% The acknowledgments are automatically included only in the final and preprint versions of the paper.
\acknowledgments{
% If a paper is accepted, the final camera-ready version will (and probably should) include acknowledgments. All acknowledgments go at the end of the paper, including thanks to reviewers who gave useful comments, to colleagues who contributed to the ideas, and to funding agencies and corporate sponsors that provided financial support.
% We thank \kr{TBA}. 
We thank Field AI for providing compute resources to support this research project. 
We thank all members of the Robot Learning Lab at Stony Brook University for support, feedback and guidance. 
}

%===============================================================================
\clearpage
% no \bibliographystyle is required, since the corl style is automatically used.
\section{Additional Details}

\begin{table}[t]
\centering\small
\caption[Additional Experiments with LSS]{
We report top-1 (\%) accuracy on the Kinetics-400 \cite{kinetics400} validation set for linear probing evaluation (left). All models are pre-trained on the training set of Kinetics-400 dataset. We also report a CLIP baseline for comparison purposes. Performance of our proposed approach is on-par with prior state-of-the-art and showcases improvements over our baseline method. We also report retrieval scores (top-right) for MSR-VTT and classification mAP (bottom-right) for Charades dataset.}
\vspace{0.5em}
\begin{minipage}{0.49\textwidth}
    \centering
    \setlength{\tabcolsep}{4pt}
    \scalebox{0.90}{
    \begin{tabular}{l|c|c}
    \toprule
    \rowcolor{Gray} 
    Method                                              & Backbone & Acc (\%) \\ \midrule
    CVRL \cite{qian2020spatiotemporal} \venue{(CVPR’21)}& R3D-101  & 67.6  \\
    BraVe \cite{recasens2021broaden} \venue{(ICCV'21)}  & R3D-50   & 66.7  \\ 
    Vi$^2$CLR \cite{Diba_2021_ICCV} \venue{(ICCV '21)}  & S3D      & 63.4  \\ 
    CORP \cite{Hu_2021_ICCV}        \venue{(ICCV '21)}  & R3D-50   & 66.6  \\ 
    SVT \cite{Ran2021SVT} \venue{(CVPR `22)}                              & ViT-B    & 68.1  \\ 
    VideoMAE \cite{Tong2022VidMAE} \venue{(NeurIPS `22)}                      & ViT-B    & 61.3  \\ 
    CLIP \cite{radford2021clip}                         & ViT-B    & 66.4  \\ \midrule
    LSS (ours)                                          & ViT-B    & 67.3  \\ 
    \bottomrule
    \end{tabular}}       
    \label{lss_tbl:sota_k400}
\end{minipage}
%
\begin{minipage}{0.49\textwidth}
    \centering
    \begin{tabular}{l|c|c|c} 
    \toprule \rowcolor{Gray} 
    Method & R@1  & R@5  & R@10 \\ \midrule
    CLIP \cite{radford2021clip}  & 30.6 & 54.4 & 64.3 \\
    LSS (ours)    & 33.8 & 58.2 & 70.3 \\ \bottomrule
    \end{tabular}
    
    \vspace{2em}
    
    \begin{tabular}{l|c}
    \toprule \rowcolor{Gray} 
    Method & Classification mAP \\ \midrule
    CLIP \cite{radford2021clip}  & 19.7         \\
    LSS (ours)  & 23.1         \\ \bottomrule
    \end{tabular}
\end{minipage}

\end{table}


\subsection{Prompting details}
\label{app:gpt_prompting}
Our proposed approach utilizes two sets of language based captions: categories and descriptions. While categories are obtained directly from the class labels of datasets (set of unique labels - e.g. 400 classes in Kinetics-400 dataset), the descriptions are generated automatically utilizing GPT-3 \cite{brown2020language}. For each category caption, we query GPT-3 to provide a set of descriptions and visual characteristics. 

In detail, we use the following two prompts to generate descriptions and visual characteristics:
\\
\texttt{prompt1 = "Give 4 different descriptions for the phrase: \{category\}?"} \\
\texttt{prompt2 = "List visual objects or characteristics usually seen with the action: \{category\}?"}
\\
The resulting two sets of captions are converted to text embeddings using our text-encoder, and a single average text embedding is computed. This averaged embedding is used as the description basis vector for that category. Also, the resulting dataset containing these category-description pairs is made available publicly. 


\subsection{Additional Experiments}

\noindent \textbf{Linear Probing Evaluation:}
We present more results for linear probing in \cref{lss_tbl:sota_k400} (left). Our proposed LSS improves over the baseline achieving competitive performance on Kinetics-400. 

\noindent \textbf{Text-to-video retrieval:}
An important characteristic of CLIP \cite{radford2021clip} is its retrieval ability across both language and visual modalities. In order to verify if proposed LSS retains these strengths, we run experiments on MSR-VTT text-to-video retrieval benchmark. We demonstrate how LSS improves over our baseline CLIP, reporting these results in \cref{lss_tbl:sota_k400} (top-right). 

\noindent \textbf{Charades Evaluation:}
 We explore an alternate task of zero-shot multi-label classification on the Charades video dataset. We report mAP results for this task in \cref{lss_tbl:sota_k400} (bottom-right) as an additional point of comparison.


\bibliography{main}  % .bib



\end{document}
