% related_work_thesis.tex — Condensed related work for lit review chapter
% Full history of illusions section is in history_thesis.tex (included in chapter body)

\section{Illusion Generation and Score Distillation}
\label{diffill_sec:related}

\textbf{Classical and Computational Illusions:} Images whose interpretation depends on viewing angle or category bias have been designed for centuries, drawing scholarly interest from psychologists \cite{jastrow1899mind, boring_new_1930} and philosophers \cite{wittgenstein_philosophical_1953}. Computational methods later automated specific illusion types: hybrid images \cite{oliva2006hybrid} combine frequency-separated features so that close and distant viewing reveal different objects, multi-view wire art \cite{hsiao_multi-view_2018} produces 3D wires that project as different line drawings from different angles, and view-dependent surfaces \cite{perroni-scharf_constructing_2023} extend this to colored height fields. Steganographic approaches such as The Magic Lens \cite{papas_magic_2012} embed hidden images recoverable through refractive optics. Our Diffusion Illusions framework generalizes across these illusion families using a unified score distillation formulation.

\textbf{Diffusion-based Image Generation:} Diffusion Probabilistic Models \cite{pmlr-v37-sohl-dickstein15} resulted in rapid advances for image generation tasks, including text-to-image generation \cite{Nichol2022GLIDETP,diffusionbeatsgans,dalle,dalle2,imagen,palette,parti,sr3}. Recent works \cite{Poole2022DreamFusionTU,Burgert2022PeekabooTT} sample pre-trained diffusion models without re-training to generate outputs in novel domains. Score Distillation introduced in DreamFusion \cite{Poole2022DreamFusionTU} is the underlying technique enabling optimization of samples in any arbitrary parameter space without backpropagation through the diffusion model. We utilize these techniques to construct a novel framework for illusion generation. These rapid advances have led to an exploration of suitable evaluation metrics, both quantitative and qualitative \cite{lee_holistic_2023,Benny2020EvaluationMF,Betzalel2022ASO,yeh2023navigating, friedman2022vendi}, which we use to evaluate our proposed framework.

\textbf{Contemporary Work:} Following recent image generation developments, a small but growing body of non-scholarly or unpublished work has approached the problem of generating multi-view 2D images \cite{tancik_illusiondiffusion_2023} or ambigrams \cite{samsudin_ambigrams_2023}. While these approaches appear to yield appealing results, they are narrowly focused on specific illusions, may require substantial cherry-picking, and have not been formally presented or published. In contrast, we present a formalized, generic approach capable of generating variable types of illusions followed by extensive evaluation (both quantitative and qualitative) of our approach. Inspired by our Diffusion Illusions project, contemporary work in \cite{geng2023visualanagrams} presents a formal framework for efficient (fast inference) illusion generation, but operates on a subset of our illusions (namely, those with a single ``prime image'' in our terminology). Furthermore, \cite{geng2023visualanagrams} does not explore any illusions with overlay which is generally more challenging and the generality for real-world transfer (i.e. fabrication of illusions in the real world).
