% related_work_thesis.tex — Related work section extracted from Diffusion Illusions main.tex

\section{Illusion Generation and Score Distillation}
\label{diffill_sec:related}

\textbf{Classical and Computational Illusions:} Images whose interpretation depends on viewing angle or category bias have been designed for centuries, drawing scholarly interest from psychologists~\cite{jastrow1899mind, boring_new_1930} and philosophers~\cite{wittgenstein_philosophical_1953}. Computational methods later automated specific illusion types: hybrid images~\cite{oliva2006hybrid} combine frequency-separated features so that close and distant viewing reveal different objects, multi-view wire art~\cite{hsiao_multi-view_2018} produces 3D wires that project as different line drawings from different angles, and view-dependent surfaces~\cite{perroni-scharf_constructing_2023} extend this to colored height fields. Steganographic approaches such as The Magic Lens~\cite{papas_magic_2012} embed hidden images recoverable through refractive optics. Our Diffusion Illusions framework generalizes across these illusion families using a unified score distillation formulation.

\textbf{Diffusion-based Generation and Score Distillation:} Diffusion Probabilistic Models~\cite{pmlr-v37-sohl-dickstein15} led to rapid advances in text-to-image generation~\cite{Nichol2022GLIDETP,diffusionbeatsgans,dalle,dalle2,imagen,palette,parti,sr3}. Score distillation, introduced in DreamFusion~\cite{Poole2022DreamFusionTU}, enables optimization of samples in arbitrary parameter spaces against a frozen diffusion model without backpropagation through its weights. Peekaboo~\cite{Burgert2022PeekabooTT} applies this to alpha masks for segmentation. We extend score distillation to multi-view illusion generation, optimizing pixel-space images under multiple geometric transformations simultaneously.

\textbf{Contemporary Work:} Following recent image generation developments, a small body of work has approached multi-view 2D image generation~\cite{tancik_illusiondiffusion_2023} and ambigrams~\cite{samsudin_ambigrams_2023}, but these are narrowly focused and lack formal evaluation. Inspired by our project, Visual Anagrams~\cite{geng2023visualanagrams} presents a framework for efficient illusion generation, but operates on a subset of our illusion types and does not explore overlay illusions or real-world fabrication.
