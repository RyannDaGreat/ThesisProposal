\begin{table}[t]
\centering
\small
\def\arraystretch{1.8}  % height
\setlength\tabcolsep{0.7em}  % width
\scalebox{0.86}{
\begin{tabular}{cccc}
    \toprule \rowcolor{Gray}
    Illusion & $n$ & $m$ & $\mathbf{a}$ \\ \midrule
    Flip & 1 & 2 & \makecell{$a_1(\mathbf{p}) = p_1$ \\ $a_2(\mathbf{p}) = \mathrm{rot}(p_2, 180)$} \\ \midrule
    Rotation Overlay & 2 & 4 & $a_j(\mathbf{p}) = p_1 * \mathrm{rot}(p_2, 90j)$ \\ \midrule
    Hidden Overlay & 4 & 5 & \makecell{$a_j(\mathbf{p}) = p_j, j \le 4$ \\ $a_5(\mathbf{p}) = p_1 * p_2 * p_3 * p_4$} \\ \bottomrule
\end{tabular}
}
\vspace{-0.5em}
\caption{This table describes our mathematical models of the Flip, Rotation Overlay, and Hidden Overlay illusions, describing the number of prime images $n$, the number of derived images $m$, and the arrangement operator $\mathbf{a}$ mapping from prime image space $\mathcal{P}^n$ to derived image space $\mathcal{D}^m$. The arrangements in the Flip illusion are simply the identity and a 180 degree rotation. The arrangement operations in the Overlay illusions use a multiplication blend operation to model shining light through multiple transparencies; the result is multiplied by a constant and normalized using $\mathrm{tanh}$ to avoid losing dynamic range.}
\label{tbl:arrangements}
\vspace{-0.5em}
\end{table}

%% We can revert back to old format using this if preferred 
% \begin{tabular}{|c|c|c|c|}
%     \hline
%     Illusion & $n$ & $m$ & $\mathbf{a}$ \\
%     \hhline{|=|=|=|=|}
%     Flip & 1 & 2 & \makecell{$a_1(\mathbf{p}) = p_1$ \\ $a_2(\mathbf{p}) = \mathrm{rot}(p_2, 180)$} \\
%     \hline
%     Rotation Overlay & 2 & 4 & $a_j(\mathbf{p}) = \mathrm{tanh}\left(2 * p_1 * \mathrm{rot}(p_2, 90j)\right)$ \\
%     \hline
%     Multiple Overlay & 4 & 5 & \makecell{$a_j(\mathbf{p}) = p_j, j \le 4$ \\ $a_5(\mathbf{p}) = \mathrm{tanh}\left(3 * p_1 * p_2 * p_3 * p_4\right)$} \\
%     \hline
% \end{tabular}