\begin{figure}
\centering
\includegraphics[width=1\linewidth]{figs/relatedwork.v3.pdf}
\vspace{-2.0em}
\caption{\small
A brief history of illusions. \textbf{Classical illusions}: (A) ``Fruit Basket'' (1500s) by Giuseppe Arcimboldo provides a very early example, depicting a face when viewed in one orientation and a fruit basket when viewed in the other. (B) When viewed directly, ``Kaninchen und Ente'' (1892) is ambiguous; $45^\circ$ rotations make it appear as a rabbit or a duck \cite{jastrow1899mind, wittgenstein_philosophical_1953}. (C) ``My Wife and My Mother-in-Law'' (1915) by William Ely Hill may be interpreted as showing either a young or an old woman depending on how it is grouped \cite{boring_new_1930, nicholls_perception_2018}. \textbf{Computationally-generated illusions}: (D) a hybrid image which appears to be a leopard when viewed close-up and an elephant when viewed from a distance \cite{oliva2006hybrid}. (E) a wire sculpture which depicts three different 2010s American politicians when viewed from different angles \cite{hsiao_multi-view_2018}. \textbf{Diffusion-based illusions}: (F) an image depicting a duck when viewed upright and a rabbit when rotated $90^\circ$ ccw \cite{tancik_illusiondiffusion_2023}. (G) a set of computationally-generated ambigrams reading `Ohio', `cloud', `yeah', and `python' \cite{samsudin_ambigrams_2023}. (H) an image depicting a giraffe when viewed upright and a penguin when viewed upside-down \cite{geng2023visualanagrams}.}
\label{fig:intro}
\vspace{-1em}
\end{figure}