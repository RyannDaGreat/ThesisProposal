\begin{figure*}[t]

\begin{minipage}{\linewidth}
	\includegraphics[width=0.195\linewidth]{figures__vis_appendix__001.png}
	\includegraphics[width=0.195\linewidth]{figures__vis_appendix__002.png}
	\includegraphics[width=0.195\linewidth]{figures__vis_appendix__003.png}
	\includegraphics[width=0.195\linewidth]{figures__vis_appendix__004.png}
	\includegraphics[width=0.195\linewidth]{figures__vis_appendix__010.png}
\end{minipage}
\begin{minipage}{\linewidth}
	\includegraphics[width=0.195\linewidth]{figures__vis_appendix__005.png}
	\includegraphics[width=0.195\linewidth]{figures__vis_appendix__006.png}
	\includegraphics[width=0.195\linewidth]{figures__vis_appendix__007.png}
	\includegraphics[width=0.195\linewidth]{figures__vis_appendix__008.png}
	\includegraphics[width=0.195\linewidth]{figures__vis_appendix__009.png}
\end{minipage}
\caption{\textbf{\modelname is truly open vocabulary}:
We illustrate examples where \modelname is able to localize various regions of interest defined by references from popular culture. All of these are examples where the model has been able to correctly localize (identified region centroid or high IoU with correct region). The caption below each image is used to generate the same color mask. 
An interesting behavior of our model is its localization to the region most defined by the accompanying text caption. For example, in the case of Emma Watson in the top row second figure, it is visible how \modelname localizes on her face and body instead of her frock (see \cref{fig:vis_ew} for more on this).
}
\label{fig:vis_more}
% 	\vspace{-1em}

\end{figure*}
