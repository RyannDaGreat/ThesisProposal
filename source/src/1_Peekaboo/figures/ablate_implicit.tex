\begin{figure*}[!h]
\centering
\begin{minipage}{\linewidth}
    \includegraphics[width=\linewidth]{figures__vis_appendix__neural_cat.jpg}\\
    \includegraphics[width=\linewidth]{figures__vis_appendix__raster_cat.jpg} \\
   \hspace*{\fill}\small{$\overrightarrow{\texttt{Iteration}}$}\hspace*{\fill} \\
\end{minipage}
\begin{minipage}{0.5\linewidth}
    \includegraphics[width=0.49\linewidth]{figures__vis_appendix__neural_graph.png}
    \includegraphics[width=0.49\linewidth]{figures__vis_appendix__raster_graph.png}
\end{minipage}
\begin{minipage}{0.49\linewidth}
    \caption{\textbf{Ablation on implicit neural representation}:
    In \modelname's \vart{Fourier} and \vart{Bilateral Fourier} variants, as well as our transparency image generations, we utilize neural-neural textures \cite{Burgert2022}, which use Fourier feature networks \cite{fourier_feature_networks}. Here we compare against the alternative of direct pixel-level representations for RGB image generation. 
    (top) The first row shows images generated using implicit representations while the second row shows those of pixel-level ones. The two graphs below correspond to pixel variance (y-axis) plotted against iteration (x-axis) for implicit and pixel-level representation respectively from left to right.}
    \label{fig:ablate_neural}
\end{minipage}
\vspace{-1em}
\end{figure*}