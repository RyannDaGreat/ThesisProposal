\section{Conclusion}
\label{sec:conclusion}

In this work, we introduce a novel and faster-than-real-time noise warping algorithm that seamlessly incorporates motion control into video diffusion noise sampling, bridging the gap between chaos and order in generative modeling. By leveraging this noise warping technique to preprocess video data for video diffusion fine-tuning, we provide a unified paradigm for a wide range of user-friendly, motion-controllable video generation applications. Extensive experiments and user studies demonstrate the superiority of our method in terms of visual quality, motion controllability, and temporal consistency, making it a robust and versatile solution for motion control in video diffusion models.

\section*{Acknowledgments}
\label{sec:acknowledgment}

We would like to express our gratitude to Stephan Trojansky and Jeffrey Shapiro for their initial and ongoing executive support; Sebastian Sylwan, Daniel Heckenberg, Jitendra Agarwal, Matheus Leão, and Sungmin Lee for their IT support; Xueming Yu and David George for their hardware support; Jennifer Lao and Lianette Alnaber for their operational support; and Winnie Lin, Ahmet Tasel, Yiqun Mei, Lukas Lepicovsky, Rahul Garg, Ashish Rastogi, Ritwik Kumar, Cornelia Carapcea, and Girish Balakrishnan for their insightful technical discussions.

\section*{Social impact statement}
\label{sec:social_impact_statement}

Our work contributes to the growing field of video generative models by advancing motion-controllable video generation, which has the potential to revolutionize creative industries such as filmmaking and animation. By introducing a computationally efficient and accessible framework, our method democratizes high-quality video generation, enabling creators, developers, and artists to produce dynamic content with minimal resources or specialized training.

However, we acknowledge the potential misuse of such technology, including the creation of deepfakes or misleading media. To mitigate these risks, we advocate for responsible use, proper content labeling, and the integration of detection mechanisms to ensure ethical deployment. Our approach also emphasizes compatibility with diverse models, encouraging transparency and collaboration within the research community to address societal concerns effectively while maximizing the positive impact of this technology.