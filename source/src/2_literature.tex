\chapter{Literature Review}
\label{chapter:literature}

Controlling what diffusion models generate has emerged as a central challenge following rapid progress in image and video synthesis. Early diffusion probabilistic models established the theoretical foundations of iterative denoising, and subsequent work on score-based generative modeling made these methods practical for high-fidelity image synthesis. The introduction of latent diffusion brought computational efficiency by operating in a compressed latent space, enabling large-scale text-to-image systems such as GLIDE, DALL-E, Imagen, and Stable Diffusion XL. A parallel line of work explored steering these models beyond text prompts: score distillation sampling showed that frozen diffusion models could supervise optimization in arbitrary parameter spaces, while adapter-based conditioning mechanisms such as ControlNet demonstrated that spatial signals like edges, depth, and poses could guide generation without full retraining. As image diffusion matured, attention shifted to the temporal domain, with video diffusion models extending these architectures to generate coherent motion---though controlling that motion remained largely unsolved.

Building on this progression, a growing body of work has sought finer-grained control over both the spatial and temporal structure of diffusion model outputs. Training-free approaches revealed that frozen models encode rich semantic structure that can be extracted through inference-time optimization alone, enabling tasks from zero-shot segmentation to multi-view illusion generation without any model retraining. On the data and conditioning side, the scarcity of large-scale alpha matting datasets has limited the ability to train models with precise transparency boundaries, and generating such datasets synthetically presents its own challenges around background selection and automated quality control. For video, the difficulty of specifying motion through text alone has motivated noise-space manipulation approaches that inject temporal structure directly into the sampling process, as well as trajectory-based editing methods that modify object and camera motion in existing footage. Despite these advances, most control methods address either spatial or temporal domains but not both, and the field lacks a unified perspective on how different control mechanisms relate to one another.

Motivated by these converging directions, this thesis investigates controlling diffusion models across a range of tasks and modalities. We develop methods that extract spatial localization from frozen diffusion models for zero-shot segmentation (Peekaboo), compose score distillation estimates under multiple geometric transformations to generate optical illusions (Diffusion Illusions), construct a large-scale alpha matte dataset through automated chroma keying of generated images (MAGICK), inject motion structure into video generation through noise warping along optical flow fields (Go-with-the-Flow), and enable true video-to-video motion editing via training on synthetic motion counterfactual pairs (MotionV2V). Each section that follows presents the detailed related work specific to its contribution.

% related_work_thesis.tex — Related work section extracted from Diffusion Illusions main.tex

\section{Illusion Generation and Score Distillation}
\label{diffill_sec:related}

\textbf{Classical and Computational Illusions:} Images whose interpretation depends on viewing angle or category bias have been designed for centuries, drawing scholarly interest from psychologists~\cite{jastrow1899mind, boring_new_1930} and philosophers~\cite{wittgenstein_philosophical_1953}. Computational methods later automated specific illusion types: hybrid images~\cite{oliva2006hybrid} combine frequency-separated features so that close and distant viewing reveal different objects, multi-view wire art~\cite{hsiao_multi-view_2018} produces 3D wires that project as different line drawings from different angles, and view-dependent surfaces~\cite{perroni-scharf_constructing_2023} extend this to colored height fields. Steganographic approaches such as The Magic Lens~\cite{papas_magic_2012} embed hidden images recoverable through refractive optics. Our Diffusion Illusions framework generalizes across these illusion families using a unified score distillation formulation.

\textbf{Diffusion-based Generation and Score Distillation:} Diffusion Probabilistic Models~\cite{pmlr-v37-sohl-dickstein15} led to rapid advances in text-to-image generation~\cite{Nichol2022GLIDETP,diffusionbeatsgans,dalle,dalle2,imagen,palette,parti,sr3}. Score distillation, introduced in DreamFusion~\cite{Poole2022DreamFusionTU}, enables optimization of samples in arbitrary parameter spaces against a frozen diffusion model without backpropagation through its weights. Peekaboo~\cite{Burgert2022PeekabooTT} applies this to alpha masks for segmentation. We extend score distillation to multi-view illusion generation, optimizing pixel-space images under multiple geometric transformations simultaneously.

\textbf{Contemporary Work:} Following recent image generation developments, a small body of work has approached multi-view 2D image generation~\cite{tancik_illusiondiffusion_2023} and ambigrams~\cite{samsudin_ambigrams_2023}, but these are narrowly focused and lack formal evaluation. Inspired by our project, Visual Anagrams~\cite{geng2023visualanagrams} presents a framework for efficient illusion generation, but operates on a subset of our illusion types and does not explore overlay illusions or real-world fabrication.


% related_work_thesis.tex — Related work section extracted from Diffusion Illusions main.tex

\section{Illusion Generation and Score Distillation}
\label{diffill_sec:related}

\textbf{Classical and Computational Illusions:} Images whose interpretation depends on viewing angle or category bias have been designed for centuries, drawing scholarly interest from psychologists~\cite{jastrow1899mind, boring_new_1930} and philosophers~\cite{wittgenstein_philosophical_1953}. Computational methods later automated specific illusion types: hybrid images~\cite{oliva2006hybrid} combine frequency-separated features so that close and distant viewing reveal different objects, multi-view wire art~\cite{hsiao_multi-view_2018} produces 3D wires that project as different line drawings from different angles, and view-dependent surfaces~\cite{perroni-scharf_constructing_2023} extend this to colored height fields. Steganographic approaches such as The Magic Lens~\cite{papas_magic_2012} embed hidden images recoverable through refractive optics. Our Diffusion Illusions framework generalizes across these illusion families using a unified score distillation formulation.

\textbf{Diffusion-based Generation and Score Distillation:} Diffusion Probabilistic Models~\cite{pmlr-v37-sohl-dickstein15} led to rapid advances in text-to-image generation~\cite{Nichol2022GLIDETP,diffusionbeatsgans,dalle,dalle2,imagen,palette,parti,sr3}. Score distillation, introduced in DreamFusion~\cite{Poole2022DreamFusionTU}, enables optimization of samples in arbitrary parameter spaces against a frozen diffusion model without backpropagation through its weights. Peekaboo~\cite{Burgert2022PeekabooTT} applies this to alpha masks for segmentation. We extend score distillation to multi-view illusion generation, optimizing pixel-space images under multiple geometric transformations simultaneously.

\textbf{Contemporary Work:} Following recent image generation developments, a small body of work has approached multi-view 2D image generation~\cite{tancik_illusiondiffusion_2023} and ambigrams~\cite{samsudin_ambigrams_2023}, but these are narrowly focused and lack formal evaluation. Inspired by our project, Visual Anagrams~\cite{geng2023visualanagrams} presents a framework for efficient illusion generation, but operates on a subset of our illusion types and does not explore overlay illusions or real-world fabrication.


\section{Alpha Matting and Synthetic Dataset Generation}
\label{magick_sec:relatedwork}

\textbf{Synthetic Segmentation Data Generation:}
Many methods have recently been proposed to synthetically generate segmentation data. Early methods utilize GANs for data synthesis.
DatasetGAN~\cite{zhang2021datasetgan} proposes to decode GAN latent codes to generate segmentation data, primarily of specifc object parts or of limited scenes like bedrooms.
BigDatasetGAN~\cite{zhang2022bigdatasetgan} produces masks for single primary objects by training a GAN on Imagenet~\cite{ILSVRC15}.

Diffusion-based model typically take a text prompt as input, then simultaneously synthesize an image along with a mask. The mask may be of a single primary object~\cite{wu2023diffumask,li2023open} or a semantic segmentation within the domain of an existing hand-labeled dataset~\cite{wu2023datasetdm,nguyen2023dataset}. Variations of this theme include generating multiple images and object masks at once~\cite{xie2023mosaicfusion} or generating the masks first and then the image~\cite{ye2023seggen}. Peekaboo~\cite{burgert2023peekaboo} generates single objects, the most closely related method to our own, but the results lack details like hair and fur. While arguments can be made against training with generated data~\cite{alemohammad2023selfconsuming,shumailov2023curse}, these works show that training with synthetic data, either alone or in conjunction with real data, yields results that are on par with or surpass the state-of-the-art set by methods trained with real data.

A drawback of these methods is they all focus on binary masks and lack transparencies and fine details such as hair or fur. Our method specifically targets generating images with alpha mattes that exhibit fine details.

\vspace{0.5em}

\textbf{Alpha Matting Datasets:}
While our method is the first to generate images with alpha masks, many alpha matting datasets already exist. Generating alpha mattes is a difficult task, requiring complicated image capture methods and/or excessive user interaction, resulting in the existing matting datasets being small in size. The early matting dataset from alphamatting.com~\cite{rhemann2009perceptually} was generated using triangulation matting, a tedious process of photographing the same object against multiple backgrounds, resulting in a dataset of 27 training images and 8 test images. This was extended to video to produce a small number of frames using stop-motion photography~\cite{erofeev2015perceptually}.

Manual extraction of the alpha matte from photographs using existing matting methods has been used to create several datasets~\cite{xu2017deep,shen2016deep,wang2021crgnn,sun2021semantic}. However, this approach is very time consuming and prone to error, generally resulting in only a few hundred objects.

Video matting datasets often use chroma keying~\cite{smith1996blue} to generate alpha data~\cite{zhang2021attention,lin2021real}. However, high quality chroma keying requires both careful setup of the green (blue) screen and lighting as well as manual post-processing to tweak parameters and manually correct or mask out mistakes.

While these methods can produce high-quality alpha mattes, they are difficult and tedious to collect. This contrasts our approach that can produce large numbers of accurate alpha mattes with the corresponding images with minimal user interaction.

\vspace{0.5em}

\textbf{Segmentation and Matting:}

Arguably, an alternative method of producing a dataset like ours would be to simply extract objects from generated images with standard segmentation/matting methods without bothering to produce them on green screens. Such generated images would contain background details and potentially other foreground objects that would need to be separated from the object.

While many segmentation datasets exist~\cite{lin2014microsoft,zhou2017scene,OpenImages,Cordts2016Cityscapes}, as do many methods for segmenting salient objects~\cite{li2014secrets,li2017instance,qin2022highly} or multiple objects~\cite{kirillov2019panoptic,he2018mask,Qi_2023_ICCV} from images, these methods would not yield the accurate alpha mattes that our method produces.  Matting methods~\cite{xu2017deep,soumyadip2020background,cai2019disentangled,lu2019indices,tang2019learning,dai2022boosting}, despite significant progress recently, are still imperfect and would include errors or artifacts in the alpha mattes. Indeed, we hope our dataset will be used to improve matting methods in future works.


\section{Motion Control in Video Diffusion Models}
\label{gwtf_sec:related_work}

\textbf{Image and Video Diffusion Models:}
With the theoretical establishments of diffusion models~\cite{song2021score,ho2020denoising,song2021denoising,karras2022elucidating} and their practical advancements~\cite{nichol2021glide,ho2022classifier}, and when sophisticated text encoders~\cite{radford2021learning} and language models~\cite{raffel2020exploring} meet diffusion models, great breakthroughs in text-to-image generation~\cite{rombach2022high,podell2023sdxl,stabilityai2023deepfloyd} have revolutionized how we digitize and create visual worlds. Building upon these, image-to-image diffusion models~\cite{brooks2023instructpix2pix,zhang2023adding,ke2024repurposing} enable image editing applications like stylization~\cite{meng2022sdedit}, relighting~\cite{he2024diffrelight}, and super-resolution~\cite{yue2024resshift,stabilityai2023deepfloyd}, expanding creativity in recreating or enhancing visual worlds.

A natural extension of image generation use cases is to cover the temporal dimension for video generation. The most cost-efficient way is to reuse the well-trained image diffusion model weights. Directly querying the above image diffusion models using random noise to generate videos frame-by-frame often struggles with temporal inconsistency, flickering, or semantic drifting. Noise warping, HIWYN~\cite{chang2024warped}, as a method for creating a sequence of temporally-correlated latent noise from optical flow while claiming spatial Gaussianity preservation, yields temporally consistent motion patterns after querying image diffusion models without further fine-tuning. To overcome its defective spatial Gaussianity preservation and undesired time complexity, we propose a novel warped noise sampling algorithm that guarantees spatial Gaussianity and runs fast enough in real time. We validate its efficacy by applying it to the training-free image diffusion models like DifFRelight~\cite{he2024diffrelight} for video relighting and DeepFloyd IF~\cite{stabilityai2023deepfloyd} for video super-resolution.

Video diffusion model training is a more costly yet more effective way for video generation~\cite{brooks2024video,chen2023videocrafter1,blattmann2023align,guo2024animatediff,blattmann2023stable,qing2024hierarchical,xing2024dynamicrafter,yang2024cogvideox}. AnimateDiff~\cite{guo2024animatediff} upgrades pre-trained image diffusion models by fine-tuning temporal attention layers on large-scale video datasets. CogVideoX~\cite{yang2024cogvideox}, a state-of-the-art open-source video diffusion model, combines spatial and temporal dimensions by encoding/decoding videos via 3D causal VAE~\cite{yu2024language} and diffusing/denoising spatiotemporal tokens via diffusion transformers~\cite{peebles2023scalable}. We use CogVideoX~\cite{yang2024cogvideox} as a base model and incorporate our warped noise sampling for motion-controllable fine-tuning. We also fine-tune on AnimateDiff~\cite{guo2024animatediff} to show our method is model-agnostic.

\vspace{0.5em}

\textbf{Motion Controllable Video Generation:}
Beyond text~\cite{guo2024animatediff,yang2024cogvideox} and image controls~\cite{guo2024sparsectrl,xing2024dynamicrafter,zhou2024storydiffusion} for video diffusion models, motion control makes video generation more interactive, dynamically targeted, and spatiotemporally fine-grained. Current approaches to motion control follow three main paradigms:

Firstly, \textit{local object motion control} is represented by object bounding boxes or masks with motion trajectories~\cite{jain2024peekaboo,yang2024direct,wang2024motionctrl,shi2024motion,wu2024draganything,namekata2024sg,qiu2024freetraj,geng2024motion}. DragAnything~\cite{wu2024draganything} allows precise object motion manipulation in images without retraining, while SG-I2V~\cite{namekata2024sg} generates realistic, continuous video from single images using self-guided motion trajectories. These serve as recent baselines for local object motion control. Our method is plug-and-play, treating diffusion models as a black box while using synthetic flows to mimic and densify object trajectories at the pixel level.

Secondly, \textit{global camera movement control} is parameterized by camera poses and trajectories~\cite{yang2024direct,he2024cameractrl,kuang2024collaborative,wang2024motionctrl,xu2024camco,wu2024cat4d} or categorized by common directional patterns like panning and tilting~\cite{guo2024animatediff,yang2024direct}. These methods introduce additional modules that accept camera parameters, trained in a supervised manner. Other approaches~\cite{yu2024viewcrafter,hou2024training} leverage rendering priors as input for camera control. Approaches like ReCapture~\cite{zhang2024recapture} enable reconfiguration of camera trajectories in given videos. Our method bypasses the need for extensive camera parameter collection, and directly generalizes new camera movements from reference videos at inference.

Lastly, \textit{motion transfer} happens from reference videos to target contexts~\cite{wang2024videocomposer,yatim2024space,geyer2023tokenflow,yin2024scalable,ku2024anyv2v,ling2024motionclone,mou2024revideo,aira2024motioncraft}. DiffusionMotionTransfer~\cite{yatim2024space} introduces a loss that maintains scene layout and motion fidelity in target videos, while MotionClone~\cite{ling2024motionclone} uses temporal attention as motion representation, streamlining motion transfer. Using them as motion transfer baselines, we demonstrate our model's flexibility in combining reference geometries with target text guidance.


\section{Video Motion Editing}
\label{mv2v_sec:relatedworks}

Diffusion models have fundamentally reshaped media generation, evolving from foundational image synthesis frameworks~\cite{ddpm2020, stablediffusion2022} to complex video dynamics~\cite{vdm2022, imagenvideo2022, makeavideo2022, bar2024lumiere}. Recent text-conditioned video models~\cite{cogvideox2024, wan, opensora2024} have further advanced the field by adopting transformer-based architectures~\cite{dit2023} for scalable denoising.


\subsection{Conditional Video Generation}

Conditional video diffusion extends base text-to-video architectures by incorporating auxiliary control signals. Inspired by the spatial conditioning of ControlNet~\cite{controlnet2023}, recent works have adapted similar mechanisms to the temporal domain~\cite{das2025, videocontrolnet2023, motioni2v2024}, enabling guidance through depth maps, motion vectors, and camera parameters.
Concurrently, video-to-video (V2V) editing methods focus on propagating edits across frames while preserving the features of the source video~\cite{tokenflow2024, fatezero2023, codef2024, pix2video2023, tuneavideo2023, text2videozero2023, cove2024}. Many such approaches leverage DDIM inversion to facilitate appearance modifications~\cite{i2vedit2024, magicedit2023, stablevideo2023}. However, these methods are fundamentally designed for local appearance changes; they struggle with non-local motion edits where the structural correspondence between frames is disrupted. When motion patterns are altered, the temporal alignment assumptions underlying these inversion-based approaches are violated.


\subsection{Motion-Guided Video Generation}

Motion control has emerged as a critical research direction, broadly categorized into trajectory-based and optical-flow-based methods. Trajectory-based approaches condition generation on point trajectories~\cite{tora2024, dragnuwa2023, draganything2024, dragavideo2023, imageconductor2024, boximator2024, i2vcontrol2024, 3dtrajmaster2024, flextraj2024, freetraj2024, trailblazer2024}, granting precise control over object paths, camera movement, and complex interactions. Conversely, optical flow-based methods~\cite{onlyflow2024, animateanything2024} utilize dense correspondence priors derived from optical flow estimators and point trackers~\cite{raft2020, tapir2023, bootstap2024, cotracker3_2024} to achieve fine-grained motion transfer.

Despite their impressive capabilities, these methods operate primarily as \textit{generators} rather than editors. Instead of modifying an input video directly, they extract attributes (e.g., optical flow) to condition the synthesis of an entirely new video. Recent trajectory-based methods~\cite{motionprompt2024, gowiththeflow2025, ati} attempt to bridge this by conditioning on single images and motion trajectories. However, while powerful for content creation, they fail to preserve the unrevealed visual context of existing videos when motion is modified. First-frame preserving methods like ReVideo~\cite{revideo2024} attempt to address this via inpainting but degrade when camera motion reveals content absent from the initial frame.

Our method addresses these limitations to enable true video-to-video motion editing. Specifically, we allow for flexible modification of object and camera trajectories while rigorously preserving the remaining video content. This approach generalizes effectively to arbitrary objects, diverse camera motions, and complex multi-element scenes.

